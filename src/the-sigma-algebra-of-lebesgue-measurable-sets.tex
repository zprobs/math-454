$m^{\star}$ does not satisfy the third measurability requirement without the weak 3w condition.
We can construct some examples to prove this.
$A, B \subset \mathbb{R}, A \cap B = \emptyset$, such that $\mstar{A \cup B} < \mstar{A} + \mstar{B}$ later in the class.

The idea to avoid this problem is to look at ``reasonable'' subsets of $\mathbb{R}$ for which this paradox disappears.

\begin{definition}[Carath\'eodory]
    $E \subseteq R$ is called (Lebesgue) measurable if $\forall A \subset \mathbb{R}$
    \[
        \mstar{A} = \mstar{A \cap E} + \mstar{A \cap E^c}
    \]
\end{definition}
\begin{remark}
    This is equivalent to Lebesgue's definition: $E$ is measurable if and only if
    \[
        \exists U \subset \mathbb{R} \text{ such that } E \subset U \text{ and } \mstar{u \setminus E} < \epsilon
    \]
    But we will discuss this later.
\end{remark}

Suppose that $A$ is measurable and $B \subset \mathbb{R}$ is any set such that $A \cap B = \emptyset$ then
\[
    \mstar{A \cup B} = \mstar{\underbrace{(A \cup B) \cap A}_{=A}} + \mstar{\underbrace{(A \cup B) \cap A^{c}}_{=B}}
\]
Going back to our counter example for $m^{\star}$ and measurability requirement 3, $A$ or $B$ would have to be unmeasurable.

Here's another observation: For $E, A \subset \mathbb{R}$ arbitrary sets we have
\[
    A = (A \cap E) \cup (A \cap E^{c})
\]
So by 3w $\mstar{A} \leq \mstar{A \cap E} + \mstar{A \cap E^{c}}$, so $E$ is measurable $\iff$ $\forall A \subset \mathbb{R}$
\[
    \boxed{\mstar{A} \geq \mstar{A \cap E} + \mstar{A \cap E^{c}}}
\]
This holds trivially for $\mstar{A} = \infty$

\underline{Example 1:} $\emptyset$ is measurable.
$\forall A \subset \mathbb{R}$
\[
    \mstar{A} = \cancel{\mstar{A \cap \emptyset}} + \mstar{A \cap \mathbb{R}}
\]

\underline{Example 2:} $\mathbb{R}$ is measurable.
$\forall A \subset \mathbb{R}$
\[
    \mstar{A} = \mstar{A \cap \mathbb{R}} + \mstar{A \cap E^{c}}
\]

\begin{prop}
    $E \subset \mathbb{R}$ with $\mstar{E} = 0$, then $E$ is measurable.
\end{prop}

\begin{*corollary}
    Every countable set is measurable.
    $\mathbb{Q}$ measurable $\rightarrow \mathbb{R} \setminus \mathbb{Q}$ are measurable
\end{*corollary}

\begin{proof}
    Let $A \subset \mathbb{R}$ be any set
    \begin{align*}
        A \cap E \subset E &\Rightarrow \mstar{A \cap E} \leq \mstar{E} = 0 \\
        A \cap E^{c} \subset A &\Rightarrow \mstar{A \cap E^{c}} \leq \mstar{A} \\
        \text{ So } \mstar{A} &\geq \mstar{A \cap E^c} + \cancel{\mstar{A \cap E}}
    \end{align*}
\end{proof}

Our goal is to show that Lebesgue measurable sets $\mathcal{L} = \{ E \subset \mathbb{R} \mid E \text{ is measurable} \}$ is a $\siga$ on $\mathbb{R}$.
We just need to show that if $\{ E_j \}_{j=1}^{\infty}$ with $E_j \in \mathcal{L}, \; \forall j$, then $\cup_{j=1}^{\infty} E_j \in \mathcal{L}$

\begin{prop}
    If $\{ E_j \}_{j=1}^{n} \subset \mathcal{L}$ then $\cup_{j=1}^{n} E_i \in \mathcal{L}$
\end{prop}

\begin{proof}
    We use mathematical induction.
    $n=1$ is trivial so we set the base case as $n=2$.
    $E_1, E_2$ are measurable, Let $A \subset \mathbb{R}$ be any set
    \begin{align*}
        \mstar{A} &= \mstar{E_1 \cap A} + \mstar{A \cap E_1 ^{c}} \\
        &= \mstar{A \cap E_1} + \mstar{(A \cap E_1^{c}) \cap E_2} + \mstar{(A \cap E_1 ^{c}) \cap E_2 ^{c}} \\
        &= \mstar{A \cap E_1} + \mstar{(A \cap E_1^{c}) \cap E_2} + \mstar{A \cap (E_1 ^{c} \cap E_2 ^{c})} \\
        &= \mstar{A \cap E_1} + \mstar{(A \cap E_1^{c}) \cap E_2} + \mstar{A \cap (E_1 \cup E_2)^{c}}\\
        &\geq \mstar{A \cap (E_1 \cup E_2)} + \mstar{A \cap (E_1 \cup E_2)^{c}} \tag{3w} \\
    \end{align*}
    So $E_1 \cup E_2 \in \mathcal{L}$.

    Induction step $n \geq 2$
    \[
        \bigcup_{j=1}^{\infty} E_j = \left( \bigcup_{j=1}^{n-1} E_j \right) \cup E_n \in \mathcal{L} \text{ by the $n=2$ case} \qedhere
    \]
\end{proof}

To prove that this also applies to countable sets, we use
\begin{prop}[Analog of measurability requirement 3 for $m^{\star} \mid \mathcal{L}$]
    Suppose $A \subset \mathbb{R}$ is any set and $\{ E_j \}_{j=1}^{n}$ is a finite disjoint collection of sets $E_j \in \mathcal{L}$, then
\[
    \mstar{A \cap \bigcup_{j=1}^{n} E_j} = \sum_{j=1}^{n} \mstar{A \cap E_j}
\]
    In particular take $A = \mathbb{R}$ to get $\mstar{\bigcup_{j=1}^{n}E_j} = \sum \mstar{E_j}$
\end{prop}

\begin{prop}
    If $\{ E_j \}_{j=1}^{\infty}$ is a countable family with $E_i \in \mathcal{L} \; \forall j$, then $\cup_{j=1}^{\infty} E_j \in \mathcal{L}$.
    In particular, $\mathcal{L}$ is a $\siga$.
\end{prop}

We would like to have the Borel sets be measurable, i.e  $\mathcal{B} \subset \mathcal{L}$.
Recall that $\mathcal{B} = \hat{\mathcal{F}}$, where $\mathcal{F} = \{ U \subset \mathbb{R} \mid U \text{ is open } \}$ and \verb!^! denotes the $\siga$.

This results follows from the measurability of intervals combined with the measurability of the union of measurable sets.

\begin{prop}
    If $I \subseteq \mathbb{R}$ is any interval, then $I$ is measurable.
\end{prop}

\begin{theorem}
    $\mathcal{L} =$ Lebesgue Measurable subsets of $\mathbb{R}$ form a $\siga$ that contains the Borel $\siga$ $\mathcal{B}$
\end{theorem}

\begin{proof}
    We already know that $\mathcal{L}$ is a $\siga$.
    If we can show that $\mathcal{L}$ contains all open sets $U \subset \mathbb{R}$, then $\mathcal{L}$ (being a $\siga$) must contain $\mathcal{B}$ which is the $\siga$ generated by open sets.
    Now if $U \subset \mathbb{R}$ is any (non empty) open set then by definition $\forall x \in U, \exists I_x \ni x $ where $I_x$ is an open interval and $I_x \subset U$.

    We want to choose $I_x$ to be the ``maximal'' such.
    So by assigning
    \[
        a_x \coloneqq \inf \{z \in \mathbb{R} \mid (z,x) \subset U \} \text{ satisfies } a_x < x
    \]
    and
    \[
        b_x \coloneqq \sup \{y \in \mathbb{R} \mid (x,y) \subset U \} \text{ satisfies } x < b_x
    \]
    so $I_x \coloneqq (a_x,b_x)$ is an open interval that contains $x$ and by construction $I_x \in U$.
    It is the largest such, in the sense that if $a_x > - \infty$ then $a_x \notin U$ and symmetrically if $b_x < \infty$ then $b_x \notin U$.

    For any $y \in I_x$, we have $y<b_x$, so there is $z > y$ such that $(x,z) \subset U$ so $y \in U$.
    Indeed, if $a_x \in U$ then since $U$ open, $\exists r > 0$ such that $(a_x - r, a_x +r) \subset U$ contradicting the definition of $a_x$.

    So $U = \cup_{x \in U} I_x$.
    It is a huge union, however if $x, x\pr \in U, x \neq x\pr$, then either $I_x \cap I_{x\pr} = \emptyset$, or if not then necessarily $I_x = I_{x\pr}$, since $I_x \cup I_{x\pr}$ is then another open interval that contains $x$ \& $x\pr$ and is a subset of $U$, so by maximality it must equal $I_x$ \& $I_{x\pr}$.
    So, throwing away all repeated $I_x$, we can write $U = \cup_{i \in I} I_x$ for some $I$ where the intervals $I_{x_i}$ are pairwise disjoint.
    By density of $\mathbb{Q} \subset \mathbb{R}$, each such interval contains a different rational number $r_i \in I_{x_i}$.
    Since $\mathbb{Q}$ is countable, $I$ is at worst countable.

    So every $U$ open is an at most countable disjoint union of open intervals.
    Since such intervals belong ot $\mathcal{L}$, and $\mathcal{L}$ is a $\siga$, it follows that every $U$ open is in $\mathcal{L}$ as desired.
\end{proof}

\begin{prop}[The $\siga \; \mathcal{L}$ is also translation invariant]
    If $E \subset \mathcal{L}$ and $x \in \mathbb{R}$ then $E + x \in \mathcal{L}$
\end{prop}

\begin{proof}
    Given any $A \subset \mathbb{R}$,
    \begin{align*}
        \mstar{A} &= \mstar{A - x} \\
        &= \mstar{(A - x) \cap E} + \mstar{(A-x) \cap E^c}  \\
        &= \mstar{A \cap E + x} + \mstar{A \cap (E+x)^c} \tag{$m^{\star}$ translation invariant }
    \end{align*}
\end{proof}

\begin{remark}
    If $A \in \mathcal{L}$ with $\mstar{A} < \infty$, and $B \subset \mathbb{R}$ is any set with $A \subset B$, then
    \[
        \mstar{B \setminus A} = \mstar{B} - \mstar{A}
    \]
\end{remark}