%! Author = zachprobst
%! Date = 2021-09-05

% Preamble
\documentclass[11pt]{article}

% Packages
\usepackage{amsmath}
\usepackage{amsfonts}
\usepackage{amsthm}
\usepackage{mathtools}
\usepackage[makeroom]{cancel}
\usepackage{tcolorbox}

% Setup
\title{Honours Analysis 3}
\author{Zachary Probst \thanks{Notes from the lectures of Valentino Tosatti}}
\setlength{\parindent}{0pt}
\setlength{\parskip}{\baselineskip}
\DeclarePairedDelimiter{\ceil}{\lceil}{\rceil}
\DeclarePairedDelimiter{\card}{\lvert}{\rvert}
\newtcolorbox{mybox}{colback=blue!5!white,colframe=blue!75!black}

\newtheorem{theorem}{Theorem}[section]
\newtheorem*{prop}{Proposition}
\newtheorem{corollary}{Corollary}[theorem]
\newtheorem*{*corollary}{Corollary}
\newtheorem{lemma}[theorem]{Lemma}
\newtheorem{definition}{Definition}[section]
\newtheorem*{remark}{Remark}

% Commands
\newcommand{\siga}{\sigma\text{-algebra}}
\newcommand{\pwr}[1]{\mathcal{P}(\mathbb{#1})}
\newcommand{\rinf}{\mathbb{R}_{\geq 0} \cup \{ + \infty \}}
\newcommand{\mstar}[1]{m^{\star}\left(#1\right)}
\newcommand{\isum}[1]{\sum_{#1=1}^{\infty}}
\newcommand{\pr}{^{\prime}}
\newcommand{\gdelta}{G_{\delta}}
\newcommand{\fsigma}{F_{\sigma}}

% Document
\begin{document}

    \maketitle

    \section{Borel Sets}\label{sec:borel-sets}
    We will work for some time on $\mathbb{R}$ exclusively.
Before beginning Measure Theory: a quick recap of Topology.

\begin{definition}[Open Set]
    A subset $U \subset \mathbb{R}$ is called \emph{open} if either $U = \emptyset$ or else
    \[
        \forall x \in U, \exists r > 0 \text{ such that } (x-r,x+r) \subset U
    \]
\end{definition}

Some examples of open sets: $\emptyset, \mathbb{R}, (a,b), (a,\infty), (-\infty, a)$.
There are many more because any union of an open set is still open and any finite intersection of open sets is open.

\begin{definition}[Closed Set]
    $F \subset \mathbb{R}$ is called \emph{closed} if $\mathbb{R} \setminus F \coloneqq F^c $ is open.

    $F$ is closed $\iff$ $F$ contains all points $x \in \mathbb{R}$ which have the property that $\forall r > 0, (x-r, x+r) \cap F \neq \emptyset$.
\end{definition}

If $F \subset \mathbb{R}$ is any set, the closure of $F$, denoted by $\overline{F}$, is the smallest closed set that contains $F$.

\begin{definition}[Compact]
    A subset $G \subset \mathbb{R}$ is \emph{compact} if given any collection $\{ U_i \}_{i \in I}$ of open sets $U_i \subset \mathbb{R}$ with $G \subset \cup_{i \in I} U_i$, there exists $J \subset I$, $J$ finite, such that $G \subset \cup_{j \in J} U_j$
\end{definition}

\begin{theorem}[Heine-Borel]
    $G \subset \mathbb{R}$ is compact $\iff$ $G$ is closed and bounded.
    To be bounded means $G \subset (a,b)$ for some $a,b \in \mathbb{R}$.
\end{theorem}

\begin{corollary}[Nested Set Theorem]
    Let $\{ F_n \}_{n=1}^{\infty}$ be a countable collection of non-empty, bounded, closed sets $F_n \subset \mathbb{R}$ with $F_{n+1} \subset F_n \forall n$, then
    \[
        \cap_{n=1}^{\infty} F_n \neq \emptyset
    \]
\end{corollary}

\begin{proof}
    Suppose $\cap_{n=1}^{\infty} F_n = \emptyset$ so let $U_n = F_{n}^{c}$ be open sets, such that $\cup_{n=1}^{\infty} U_n = \mathbb{R}$.
    We also have that $U_n \subset U_{n+1}$, since the $F_n$ were nested.
    Now $F_1$ is compact by Heine-Borel and $F_1 \subset \cup_{n=1}^{\infty} U_n \Rightarrow $ by compactness
    I can find a finite subcover of $F_1$, say $F \subset \cup_{n=1}^{N} U_n = U_N = F_N ^{c}$

    On the other hand $F_N \subset F_1$ by the nested property which implies $F_N = \emptyset$ which is a contradiction.
\end{proof}

    \section{Measure Theory}\label{sec:measure-theory}
    We want to measure the size of a set.
We will deal with a subset of $\mathbb{R}$.

It turns out that one needs to select a class of subsets of $\mathbb{R}$ that one wants to measure.
This class of subsets will have certain properties which are as follows.

\begin{definition}[$\siga$]
    A collection $\mathcal{A}$ of subsets of $\mathbb{R}$ is called a $\siga$ if it satisfies
    \begin{enumerate}
        \item $\emptyset \in \mathcal{A}$
        \item If $A \in \mathcal{A}$ then $A^c \in \mathcal{A}$
        \item If $\{ A_n \}_{n=1}^{\infty} \subset \mathcal{A}$ then $\cup_{n=1}^{\infty} A_n \in \mathcal{A}$
    \end{enumerate}
\end{definition}

Observe the following:
\begin{itemize}
    \item $\mathbb{R} \in \mathcal{A}$ always
    \item If $\{ A_n \}_{n=1}^{N} \subset \mathcal{A}$ then $\cup_{n=1}^{N} A_n \in \mathcal{A}$ (just define $A_n = \emptyset$ for $n > N$)
    \item If $\{ A_n \}_{n=1}^{\infty} \subset \mathcal{A}$ then $\cap_{n=1}^{\infty} A_n \in \mathcal{A}$ (since $(\cap_{n=1}^{\infty} A_n)^c = \cup_{n=1}^{\infty} A_{n}^{c}$)
    \item If $A,B \in \mathcal{A}$ then $A \setminus B \in \mathcal{A}$ too since $A \setminus B = A \cap B^c$
\end{itemize}

\underline{\textbf{Examples}:}

\begin{enumerate}
    \item $\mathcal{A} = \{ \emptyset, \mathbb{R} \}$ ``Minimal $\siga$''
    \item $\mathcal{A} = \mathcal{P}(\mathbb{R}) = $ Collection of all subsets of $\mathbb{R}$.
    ``Maximum $\siga$''
\end{enumerate}

In fact, if $\mathcal{A}$ is any $\siga$, then $\{ \emptyset,\mathbb{R} \} \subseteq \mathcal{A} \subseteq \mathcal{P}(\mathbb{R})$

For better examples, let $F$ be any collection of subsets of $\mathbb{R}$.
I want to make $F$ into a $\siga$.
Define $m = \{ \mathcal{A} \mid \mathcal{A} \text{ is a } \siga \text{ that satisfies } F \subset \mathcal{A} \}$.
$m \neq \emptyset$ since it contains $\mathcal{P}(\mathbb{R})$

If $\mathcal{A}, \mathcal{B} \in m$, I can define $\mathcal{A} \cap \mathcal{B} = \{ A \subset \mathbb{R} \mid A \in \mathcal{A} \text{ and } A \in \mathcal{B} \}$ and I can do the same for $\cap_{i \in I} \mathcal{A}$ arbitrary intersection of $\siga$ is still a $\siga$

Define $\hat{F_i} = \cap_{\mathcal{A} \in m} \mathcal{A}$ as a $\siga$ and $F \subset \hat{F}$ and it is the minimal $\siga$ with these properties.
If $G$ is a $\siga$ with $F \subset G$, then $\hat{F} \subset G$.
$\hat{F}$ is the $\siga$ generated by $F$.
Concretely, $\hat{F}$ consists of all subsets of $\mathbb{R}$ that can be constructed by applying countable unions, intersections, and complements to elements of $F$.

\begin{definition}[Borel Sets]
    The $\siga \; \mathcal{B}$ of Borel Sets is the $\siga \; \hat{F}$ generated by
    \[
        F = \{ U \subset \mathbb{R} \mid U \text{ open } \}
    \]
\end{definition}

\begin{remark}
    $\mathcal{B}$ is also the $\siga$ generated by the family of all closed subsets of $\mathbb{R}$
\end{remark}

Singletons $\{ x \} \subset \mathbb{R}$ are closed so if $A \subset \mathbb{R}$ is at most countable then $A$ is Borel.
(e.g $\mathbb{Q} \subset \mathbb{R}$) (e.g $\mathbb{R} \setminus \mathbb{Q}$)

Not all Subsets of $\mathbb{R}$ are Borel.
One can actually show that the cardinality of $\mathcal{B}$ is the same as the cardinality of $\mathbb{R}$.
On the other hand $\mathcal{P}(\mathbb{R})$ has strictly larger cardinality.

    \section{Lebesgue Outer Measure}\label{sec:lebesgue-outer-measure}
    We are hoping to measure the size of subsets of $\mathbb{R}$.
Ideally we would like to find or construct a function
\[
    m: \pwr{R} \rightarrow \mathbb{R}_{\geq 0} \cup \{ + \infty \} = [0, \infty]
\]
Which satisfies the following measure requirements:
\begin{enumerate}
    \item If $I=[a,b]$ or $(a,b)$ or $[a,b)$, or $(a,b]$, $a,b \in \mathbb{R}, a \leq b$ then $m(I) = b-a = $ measure of interval
    \item $m$ is translation invariant.
    i.e if $E \subset \mathbb{R}$ and $x \in \mathbb{R}$, let $E + x = \{ y+x \mid y \in E \}$ then $m(E+x)$ = m(E)
    \item If $\{ E_j \}_{j=1}^{n}$ is a finite collection of pairwise disjoint $E_j \subset \mathbb{R}$ then
    \[
        m \left( \cup_{j=1}^{n} E_j \right) = \sum_{j=1}^{n} m(E_j)
    \]
    \item The same as (3) except for $n = \infty$
\end{enumerate}

\begin{theorem}
    There is no such $m$ satisfying all 4 requirements
\end{theorem}

The proof for this will come later.
The solution for this is that we do not try to measure all subsets of $\mathbb{R}$.
So we have $m: \pwr{R} \rightarrow [0, \infty]$ but now we will just be happy with $m: \mathcal{A} \rightarrow [0,\infty]$ where $\mathcal{A}$ is a $\siga$ which has enough elements.
For example $\mathcal{A} > \mathcal{B}$.

We will follow H. Lebesgue as we proceed in two steps.

\underline{Step 1:} construct Lebesgue outer measure $m^{\star}: \pwr{R} \rightarrow [0, \infty]$ satisfying requirements 1,2, and 3.

\underline{Step 2:} Use $m^{\star}$ to define $\mathcal{A}$ and let $m \subset m^{\star} \mid \mathcal{A}$

To create this Lebesgue outer measure on $\mathbb{R}$ we satisfy a weakened version of requirement (3) that can be called (3w).
For any countably infinite collection $\{ E_j \}_{j=1}^{\infty}$ of arbitrary subsets $E_j \subset \mathbb{R}$
\[
    m^{\star}(\cup_{j=1}^{\infty} E_j) \leq \sum_{j=1}^{\infty} m(E_j)
\]
\begin{theorem}[Lebesgue Outer Measure]
    There is a map $m^{\star}: \pwr{R} \rightarrow \rinf $ that satisfies the measure requirements 1, 2, and 3w.
\end{theorem}

This $m^{\star}$ is called the Lebesgue outer measure on $\mathbb{R}$.

How do we define outer measure $m^{\star}(A)$?

Observe that any $A \subseteq \mathbb{R}$ can be covered by some countable infinite collection $\{ I_j \}_{j=1}^{\infty}$ of bounded open intervals, which are allowed to be empty, but we do not assume that $I_j$ be pairwise disjoint.

For example: $I_j = (-j, j), \; j=1,2,3\hdots$

Let
\[
    \mathcal{C}_A = \{ \{I_j\}_{j=1}^{\infty} \mid I_j \text{ bounded open intervals such that } A \subset \cup_{j=1}^{\infty} I_j \}
\]
$\mathcal{C}_A \neq \emptyset$ by our example so for each $\{ I_j \} \in \mathcal{C}_A$, I can consider
\[
    \sum_{j=1}^{\infty} \ell(I_j) \in \rinf \tag{$\ell$ denotes length}
\]

\begin{definition}[Outer Measure]
    \[
        \boxed{m^{\star}(A) \coloneqq \inf_{\{ I_j \} \in \mathcal{C_A}} \sum_{j=1}^{\infty} \ell (I_j)  } \in \rinf
    \]
\end{definition}

This defines a map $m^{\star}: \pwr{R} \rightarrow \rinf$

\underline{Simple Properties:}

\begin{itemize}
    \item \emph{Monotonicity}:
    If $A \subseteq B$ then $m^{\star}(A) \leq m^{\star}(B)$.
    Indeed by definition $\mathcal{C}_B \subseteq \mathcal{C}_A$ hence the infimum over $\mathcal{C}_B$ is $\geq$ than the infimum over $\mathcal{C}_A$.
    \item \emph{Empty Set}: $\mstar{\emptyset} = 0$.
    Given any $1 > \epsilon > 0$, let $I_j = (-\epsilon^{j}, \epsilon^j), \; j=1,2,\hdots$
    $\{I_j\} \in \mathcal{C}_{\emptyset}$ and $\sum_{j=1}^{\infty} \ell (I_j) = 2 \sum_{j=1}^{\infty} \epsilon^j = \frac{2\epsilon}{1-\epsilon} $ from the geometric series going to zero so $\mstar{\emptyset} \leq \frac{2\epsilon}{1-\epsilon} \; \forall 0 < \epsilon < 1$
    \item If $A \in \mathbb{R}$ is finite or countable infinite then $\mstar{A} = 0$.
    Indeed enumerate all elements of $A$ by $\{ a_j \}_{j=1}^{\infty}$.
    (If $A$ is finite say $\card{A} = n$ let $a_j = a_n$ for all $j > n$).
    For any $0 < \epsilon < 1$, let $I_j = \left( -\epsilon^j + a_j, a_j + \epsilon^j \right)$ so $A \subseteq \cup_{j=1}^{\infty} I_j$ and $\isum{j} \ell (I_j) = \frac{2\epsilon}{1-\epsilon}$ hence as before, $\mstar{A} = 0$.
    For example $\mstar{\mathbb{Q}} = 0$
\end{itemize}

We will now prove that the Lebesgue outer measure satisfies 1, 2, and 3w of the measure requirements.

\underline{Proof of Property 1}:
i.e $\mstar{I} = \ell(I)$ for any interval $I \subseteq \mathbb{R}$

Assume that $I = [a,b]$, $a < b$ are finite numbers.
Assume that $I$ is a bounded closed interval.
Our goal is to show that $\mstar{I} = b -a$.
One direction of inequality is easy to prove, the other is quite tedious and will be left out.

For any $\epsilon > 0$ let $I_1 = (a-\epsilon, b + \epsilon) > I$, let $I_j = \emptyset, j \geq 2$ so $\{I_j \} \in \mathcal{C}_I \Rightarrow \mstar{I} \leq \isum{j} \ell (I_j) = b - a + 2 \epsilon$.
Let $\epsilon \rightarrow 0$ and we obtain $\mstar{I} \leq b-a$.

\underline{Proof of Property 2}:
i.e $\forall A \subset \mathbb{R}, \forall x \in \mathbb{R}$, $\mstar{A+x} = \mstar{A}$

$\mathcal{C}_A$ and $\mathcal{C}_{A+x}$ are naturally in bijection via $\{ I_j \} \leftrightarrow \{ I_j + x \}$.
Furthermore $\ell (I_j + x) = \ell (I_j)$
\begin{align*}
    \mstar{A+x} &= \inf_{\{I_j +x\} \in \mathcal{C}_{A+x}} \isum{j} \ell (I_j + x) \\
    &= \inf_{\{I_j\} \in \mathcal{C}_{A}} \isum{j} \ell (I_j) = \mstar{A}
\end{align*}

\underline{Proof of Property 3w}:
i.e If $\{ E_j \}_{j=1}^{n}$ is a finite collection of pairwise disjoint $E_j \subset \mathbb{R}$ then $\mstar{\cup_{j=1}^{n} E_j} \leq \sum_{j=1}^{n} \mstar{E_j}$

If $\mstar{E_j} = +\infty$ for some $j$, then the property holds.
We may assume that $\mstar{E_j} < + \infty \; \forall j$.
Let $\epsilon > 0$.
By the definition of infimum, for each $j \geq 0$, there is
\[
    \{ I_{j,k} \}_{k=1}^{\infty} \in \mathcal{C}_{E_j} \text{ such that } \isum{k} \ell (I_{j,k}) < \mstar{E_j} + \epsilon 2^{-j}
\]
Thus $\{ I_{j,k} \}_{k=1}^{\infty}$ is still countable and it covers $\cup_{j=1}^{\infty} E_j$ meaning it belongs to $\mathcal{C}_{\cup_{j=1}^{\infty}} E_j$, so by definition
\[
   \mstar{\bigcup_{j=1}^{\infty} E_j} \leq \sum_{j=1}^{\infty} \isum{k} \ell (I_{j,k}) < \sum_{j=1}^{\infty} (\mstar{E_j} + \epsilon 2^{-j}) = \sum_{j=1}^{\infty} \mstar{E_j} + \epsilon
\]
Then let $\epsilon \rightarrow 0$.
Clearly, by taking all $E_j = \emptyset$ except finitely many, we have the same subadditivity 3w for finite collections.

\begin{corollary}
    $\mstar{[0,1] \cap (\mathbb{R} \setminus \mathbb{Q})} = 1 = \ell([0,1])$
\end{corollary}
\begin{proof}
    \begin{align*}
        \mstar{[0,1] \cap (\mathbb{R} \setminus \mathbb{Q})} &\leq \mstar{[0,1]} = 1 \\
        &\leq \mstar{[0,1] \cap (\mathbb{Q})} + \mstar{[0,1] \cap (\mathbb{R} \setminus \mathbb{Q})} \\
        &\leq 0 + 1
    \end{align*}
\end{proof}

\begin{corollary}
    $\mathbb{R} \setminus \mathbb{Q}$ is uncountable
\end{corollary}
\begin{proof}
    If not, then
    \[
        \mstar{\mathbb{R} \setminus \mathbb{Q}} = 0 \geq \mstar{[0,1] \cap (\mathbb{R} \setminus \mathbb{Q})} = 1 \qedhere
    \]
\end{proof}

    \section{The $\sigma$-Algebra Of Lebesgue Measurable Sets}\label{sec:the-$sigma$-algebra-of-lebesgue-measurable-sets}
    $m^{\star}$ does not satisfy the third measurability requirement without the weak 3w condition.
We can construct some examples to prove this.
$A, B \subset \mathbb{R}, A \cap B = \emptyset$, such that $\mstar{A \cup B} < \mstar{A} + \mstar{B}$ later in the class.

The idea to avoid this problem is to look at ``reasonable'' subsets of $\mathbb{R}$ for which this paradox disappears.

\begin{definition}[Carath\'eodory]
    $E \subseteq R$ is called (Lebesgue) measurable if $\forall A \subset \mathbb{R}$
    \[
        \mstar{A} = \mstar{A \cap E} + \mstar{A \cap E^c}
    \]
\end{definition}
\begin{remark}
    This is equivalent to Lebesgue's definition: $E$ is measurable if and only if
    \[
        \exists U \subset \mathbb{R} \text{ such that } E \subset U \text{ and } \mstar{U \setminus E} < \epsilon
    \]
    But we will discuss this later.
\end{remark}

Suppose that $A$ is measurable and $B \subset \mathbb{R}$ is any set such that $A \cap B = \emptyset$ then
\[
    \mstar{A \cup B} = \mstar{\underbrace{(A \cup B) \cap A}_{=A}} + \mstar{\underbrace{(A \cup B) \cap A^{c}}_{=B}}
\]
Going back to our counter example for $m^{\star}$ and measurability requirement 3, $A$ or $B$ would have to be unmeasurable.

Here's another observation: For $E, A \subset \mathbb{R}$ arbitrary sets we have
\[
    A = (A \cap E) \cup (A \cap E^{c})
\]
So by 3w $\mstar{A} \leq \mstar{A \cap E} + \mstar{A \cap E^{c}}$, so $E$ is measurable $\iff$ $\forall A \subset \mathbb{R}$
\[
    \boxed{\mstar{A} \geq \mstar{A \cap E} + \mstar{A \cap E^{c}}}
\]
This holds trivially for $\mstar{A} = \infty$

\underline{Example 1:} $\emptyset$ is measurable.
$\forall A \subset \mathbb{R}$
\[
    \mstar{A} = \cancel{\mstar{A \cap \emptyset}} + \mstar{A \cap \mathbb{R}}
\]

\underline{Example 2:} $\mathbb{R}$ is measurable.
$\forall A \subset \mathbb{R}$
\[
    \mstar{A} = \mstar{A \cap \mathbb{R}} + \mstar{A \cap E^{c}}
\]

\begin{prop}
    $E \subset \mathbb{R}$ with $\mstar{E} = 0$, then $E$ is measurable.
\end{prop}

\begin{*corollary}
    Every countable set is measurable.
    $\mathbb{Q}$ measurable $\rightarrow \mathbb{R} \setminus \mathbb{Q}$ are measurable
\end{*corollary}

\begin{proof}
    Let $A \subset \mathbb{R}$ be any set
    \begin{align*}
        A \cap E \subset E &\Rightarrow \mstar{A \cap E} \leq \mstar{E} = 0 \\
        A \cap E^{c} \subset A &\Rightarrow \mstar{A \cap E^{c}} \leq \mstar{A} \\
        \text{ So } \mstar{A} &\geq \mstar{A \cap E^c} + \cancel{\mstar{A \cap E}}
    \end{align*}
\end{proof}

Our goal is to show that Lebesgue measurable sets $\mathcal{L} = \{ E \subset \mathbb{R} \mid E \text{ is measurable} \}$ is a $\siga$ on $\mathbb{R}$.
We just need to show that if $\{ E_j \}_{j=1}^{\infty}$ with $E_j \in \mathcal{L}, \; \forall j$, then $\cup_{j=1}^{\infty} E_j \in \mathcal{L}$

\begin{prop}
    If $\{ E_j \}_{j=1}^{n} \subset \mathcal{L}$ then $\cup_{j=1}^{n} E_i \in \mathcal{L}$
\end{prop}

\begin{proof}
    We use mathematical induction.
    $n=1$ is trivial so we set the base case as $n=2$.
    $E_1, E_2$ are measurable, Let $A \subset \mathbb{R}$ be any set
    \begin{align*}
        \mstar{A} &= \mstar{E_1 \cap A} + \mstar{A \cap E_1 ^{c}} \\
        &= \mstar{A \cap E_1} + \mstar{(A \cap E_1^{c}) \cap E_2} + \mstar{(A \cap E_1 ^{c}) \cap E_2 ^{c}} \\
        &= \mstar{A \cap E_1} + \mstar{(A \cap E_1^{c}) \cap E_2} + \mstar{A \cap (E_1 ^{c} \cap E_2 ^{c})} \\
        &= \mstar{A \cap E_1} + \mstar{(A \cap E_1^{c}) \cap E_2} + \mstar{A \cap (E_1 \cup E_2)^{c}}\\
        &\geq \mstar{A \cap (E_1 \cup E_2)} + \mstar{A \cap (E_1 \cup E_2)^{c}} \tag{3w} \\
    \end{align*}
    So $E_1 \cup E_2 \in \mathcal{L}$.

    Induction step $n \geq 2$
    \[
        \bigcup_{j=1}^{\infty} E_j = \left( \bigcup_{j=1}^{n-1} E_j \right) \cup E_n \in \mathcal{L} \text{ by the $n=2$ case} \qedhere
    \]
\end{proof}

To prove that this also applies to countable sets, we use
\begin{prop}[Analog of measurability requirement 3 for $m^{\star} \mid \mathcal{L}$]
    Suppose $A \subset \mathbb{R}$ is any set and $\{ E_j \}_{j=1}^{n}$ is a finite disjoint collection of sets $E_j \in \mathcal{L}$, then
\[
    \mstar{A \cap \bigcup_{j=1}^{n} E_j} = \sum_{j=1}^{n} \mstar{A \cap E_j}
\]
    In particular take $A = \mathbb{R}$ to get $\mstar{\bigcup_{j=1}^{n}E_j} = \sum \mstar{E_j}$
\end{prop}

\begin{prop}
    If $\{ E_j \}_{j=1}^{\infty}$ is a countable family with $E_i \in \mathcal{L} \; \forall j$, then $\cup_{j=1}^{\infty} E_j \in \mathcal{L}$.
    In particular, $\mathcal{L}$ is a $\siga$.
\end{prop}

We would like to have the Borel sets be measurable, i.e  $\mathcal{B} \subset \mathcal{L}$.
Recall that $\mathcal{B} = \hat{\mathcal{F}}$, where $\mathcal{F} = \{ U \subset \mathbb{R} \mid U \text{ is open } \}$ and \verb!^! denotes the $\siga$.

This results follows from the measurability of intervals combined with the measurability of the union of measurable sets.

\begin{prop}
    If $I \subseteq \mathbb{R}$ is any interval, then $I$ is measurable.
\end{prop}

\begin{theorem}
    $\mathcal{L} =$ Lebesgue Measurable subsets of $\mathbb{R}$ form a $\siga$ that contains the Borel $\siga$ $\mathcal{B}$
\end{theorem}

\begin{proof}
    We already know that $\mathcal{L}$ is a $\siga$.
    If we can show that $\mathcal{L}$ contains all open sets $U \subset \mathbb{R}$, then $\mathcal{L}$ (being a $\siga$) must contain $\mathcal{B}$ which is the $\siga$ generated by open sets.
    Now if $U \subset \mathbb{R}$ is any (non empty) open set then by definition $\forall x \in U, \exists I_x \ni x $ where $I_x$ is an open interval and $I_x \subset U$.

    We want to choose $I_x$ to be the ``maximal'' such.
    So by assigning
    \[
        a_x \coloneqq \inf \{z \in \mathbb{R} \mid (z,x) \subset U \} \text{ satisfies } a_x < x
    \]
    and
    \[
        b_x \coloneqq \sup \{y \in \mathbb{R} \mid (x,y) \subset U \} \text{ satisfies } x < b_x
    \]
    so $I_x \coloneqq (a_x,b_x)$ is an open interval that contains $x$ and by construction $I_x \in U$.
    It is the largest such, in the sense that if $a_x > - \infty$ then $a_x \notin U$ and symmetrically if $b_x < \infty$ then $b_x \notin U$.

    For any $y \in I_x$, we have $y<b_x$, so there is $z > y$ such that $(x,z) \subset U$ so $y \in U$.
    Indeed, if $a_x \in U$ then since $U$ open, $\exists r > 0$ such that $(a_x - r, a_x +r) \subset U$ contradicting the definition of $a_x$.

    So $U = \cup_{x \in U} I_x$.
    It is a huge union, however if $x, x\pr \in U, x \neq x\pr$, then either $I_x \cap I_{x\pr} = \emptyset$, or if not then necessarily $I_x = I_{x\pr}$, since $I_x \cup I_{x\pr}$ is then another open interval that contains $x$ \& $x\pr$ and is a subset of $U$, so by maximality it must equal $I_x$ \& $I_{x\pr}$.
    So, throwing away all repeated $I_x$, we can write $U = \cup_{i \in I} I_x$ for some $I$ where the intervals $I_{x_i}$ are pairwise disjoint.
    By density of $\mathbb{Q} \subset \mathbb{R}$, each such interval contains a different rational number $r_i \in I_{x_i}$.
    Since $\mathbb{Q}$ is countable, $I$ is at worst countable.

    So every $U$ open is an at most countable disjoint union of open intervals.
    Since such intervals belong ot $\mathcal{L}$, and $\mathcal{L}$ is a $\siga$, it follows that every $U$ open is in $\mathcal{L}$ as desired.
\end{proof}

\begin{prop}[The $\siga \; \mathcal{L}$ is also translation invariant]
    If $E \subset \mathcal{L}$ and $x \in \mathbb{R}$ then $E + x \in \mathcal{L}$
\end{prop}

\begin{proof}
    Given any $A \subset \mathbb{R}$,
    \begin{align*}
        \mstar{A} &= \mstar{A - x} \\
        &= \mstar{(A - x) \cap E} + \mstar{(A-x) \cap E^c}  \\
        &= \mstar{A \cap E + x} + \mstar{A \cap (E+x)^c} \tag{$m^{\star}$ translation invariant }
    \end{align*}
\end{proof}

\begin{remark}
    If $A \in \mathcal{L}$ with $\mstar{A} < \infty$, and $B \subset \mathbb{R}$ is any set with $A \subset B$, then
    \[
        \mstar{B \setminus A} = \mstar{B} - \mstar{A}
    \]
\end{remark}

    \section{Outer and Inner Approximation of Lebesgue Measurable Sets}\label{sec:outer-and-inner-approximation-of-lebesgue-measureable-sets}

    \begin{definition}[Gebiet-Durchshnitt]
        A subset $A \subset \mathbb{R}$ is called a $G_{\delta}$ if $A = \cap_{i=1}^{\infty} A_i$ where $A_i$ are all open (possibly empty).
    \end{definition}

    \begin{definition}[Ferm\'e-Somme]
        A subset $A \subset \mathbb{R}$ is called a $F_\sigma$ if $A = \cup_{i=1}^{\infty} A_i$ where $A_i$ are all closed (possibly empty).
    \end{definition}

    Clearly, $A$ is $G_\delta \iff A^c$ is $F_\delta$.
    Also clearly, all $\gdelta$ and $\fsigma$ sets are Borel.
    Of course not all $\gdelta$ are open, e.g $[0,1] = \cap_{i=1}^{\infty} \left(- \frac{1}{i}, 1 + \frac{1}{i}\right)$ and not all $\fsigma$ are closed.
    e.g. $(0,1) = \cup_{i=1}^{\infty} \left[\frac{1}{i}, 1-\frac{1}{i}\right]$

   $\mathbb{Q}$ is clearly $\fsigma$, so $\mathbb{R} \setminus \mathbb{Q}$ is $\gdelta$.
    With this, we can give several equivalent formulations of measurability.
    
    \begin{theorem}
        Let $E \subset \mathbb{R}$ be any set, then the following are equivalent:
        \begin{enumerate}
            \item $E \in \mathcal{L}$
            \item $\forall \epsilon > 0$, $\exists U \supset E$, $U$ open, $\mstar{U \setminus E} < \epsilon$
            \item $\exists G \subset \mathbb{R}$ a $\gdelta$ set, $G \supset E$, with $\mstar{G \setminus E} = 0$
            \item $\forall \epsilon > 0$, $\exists F \subset E$, $F$ closed, $\mstar{E \setminus F} < \epsilon$
            \item $\exists F \subset \mathbb{R}$ a $\fsigma$ set, $F \subset E$ with $\mstar{E \setminus F} = 0$
        \end{enumerate}
    \end{theorem}

    \begin{prop}
        For an $E \in \mathcal{L}$ with $\mstar{E} < \infty$.
        Then $\forall \epsilon > 0$, $\exists \{ I_j \}_{j=1}^{n}$ a finite disjoint family of open intervals so that if we let $U = \cup_{j=1}^{n} I_j$ (open) then $\mstar{E \Delta U} < \epsilon$.
    \end{prop}


\end{document}