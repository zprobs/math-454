%! Author = zachprobst
%! Date = 2021-09-05

% Preamble
\documentclass[11pt]{article}

% Packages
\usepackage{amsmath}
\usepackage{amsfonts}
\usepackage{amsthm}
\usepackage{mathtools}
\usepackage[makeroom]{cancel}
\usepackage{tcolorbox}
\usepackage{centernot}
\usepackage{amssymb}
\usepackage{stmaryrd}

% Setup
\title{Honours Analysis 3}
\author{Zachary Probst \thanks{Notes from the lectures of Valentino Tosatti}}
\setlength{\parindent}{0pt}
\setlength{\parskip}{\baselineskip}
\DeclarePairedDelimiter{\ceil}{\lceil}{\rceil}
\DeclarePairedDelimiter{\card}{\lvert}{\rvert}
\newtcolorbox{mybox}{colback=blue!5!white,colframe=blue!75!black}

\newtheorem{theorem}{Theorem}[section]
\newtheorem*{prop}{Proposition}
\newtheorem{corollary}{Corollary}[theorem]
\newtheorem*{*corollary}{Corollary}
\newtheorem{lemma}[theorem]{Lemma}
\newtheorem{definition}{Definition}[section]
\newtheorem*{remark}{Remark}

% Commands
\newcommand{\siga}{\sigma\text{-algebra}}
\newcommand{\pwr}[1]{\mathcal{P}(\mathbb{#1})}
\newcommand{\rinf}{\mathbb{R}_{\geq 0} \cup \{ + \infty \}}
\newcommand{\mstar}[1]{m^{\star}\left(#1\right)}
\newcommand{\isum}[1]{\sum_{#1=1}^{\infty}}
\newcommand{\pr}{^{\prime}}
\newcommand{\gdelta}{G_{\delta}}
\newcommand{\fsigma}{F_{\sigma}}
\newcommand{\R}{\mathbb{R}}
\newcommand{\N}{\mathbb{N}}
\newcommand{\Q}{\mathbb{Q}}
\newcommand{\Le}{\mathcal{L}}
\newcommand{\intE}{\int_{E}}

% Document
\begin{document}

    \maketitle


    \section{Borel Sets}\label{sec:borel-sets}
    We will work for some time on $\mathbb{R}$ exclusively.
Before beginning Measure Theory: a quick recap of Topology.

\begin{definition}[Open Set]
    A subset $U \subset \mathbb{R}$ is called \emph{open} if either $U = \emptyset$ or else
    \[
        \forall x \in U, \exists r > 0 \text{ such that } (x-r,x+r) \subset U
    \]
\end{definition}

Some examples of open sets: $\emptyset, \mathbb{R}, (a,b), (a,\infty), (-\infty, a)$.
There are many more because any union of an open set is still open and any finite intersection of open sets is open.

\begin{definition}[Closed Set]
    $F \subset \mathbb{R}$ is called \emph{closed} if $\mathbb{R} \setminus F \coloneqq F^c $ is open.

    $F$ is closed $\iff$ $F$ contains all points $x \in \mathbb{R}$ which have the property that $\forall r > 0, (x-r, x+r) \cap F \neq \emptyset$.
\end{definition}

If $F \subset \mathbb{R}$ is any set, the closure of $F$, denoted by $\overline{F}$, is the smallest closed set that contains $F$.

\begin{definition}[Compact]
    A subset $G \subset \mathbb{R}$ is \emph{compact} if given any collection $\{ U_i \}_{i \in I}$ of open sets $U_i \subset \mathbb{R}$ with $G \subset \cup_{i \in I} U_i$, there exists $J \subset I$, $J$ finite, such that $G \subset \cup_{j \in J} U_j$
\end{definition}

\begin{theorem}[Heine-Borel]
    $G \subset \mathbb{R}$ is compact $\iff$ $G$ is closed and bounded.
    To be bounded means $G \subset (a,b)$ for some $a,b \in \mathbb{R}$.
\end{theorem}

\begin{corollary}[Nested Set Theorem]
    Let $\{ F_n \}_{n=1}^{\infty}$ be a countable collection of non-empty, bounded, closed sets $F_n \subset \mathbb{R}$ with $F_{n+1} \subset F_n \forall n$, then
    \[
        \cap_{n=1}^{\infty} F_n \neq \emptyset
    \]
\end{corollary}

\begin{proof}
    Suppose $\cap_{n=1}^{\infty} F_n = \emptyset$ so let $U_n = F_{n}^{c}$ be open sets, such that $\cup_{n=1}^{\infty} U_n = \mathbb{R}$.
    We also have that $U_n \subset U_{n+1}$, since the $F_n$ were nested.
    Now $F_1$ is compact by Heine-Borel and $F_1 \subset \cup_{n=1}^{\infty} U_n \Rightarrow $ by compactness
    I can find a finite subcover of $F_1$, say $F \subset \cup_{n=1}^{N} U_n = U_N = F_N ^{c}$

    On the other hand $F_N \subset F_1$ by the nested property which implies $F_N = \emptyset$ which is a contradiction.
\end{proof}


    \section{Measure Theory}\label{sec:measure-theory}
    We want to measure the size of a set.
We will deal with a subset of $\mathbb{R}$.

It turns out that one needs to select a class of subsets of $\mathbb{R}$ that one wants to measure.
This class of subsets will have certain properties which are as follows.

\begin{definition}[$\siga$]
    A collection $\mathcal{A}$ of subsets of $\mathbb{R}$ is called a $\siga$ if it satisfies
    \begin{enumerate}
        \item $\emptyset \in \mathcal{A}$
        \item If $A \in \mathcal{A}$ then $A^c \in \mathcal{A}$
        \item If $\{ A_n \}_{n=1}^{\infty} \subset \mathcal{A}$ then $\cup_{n=1}^{\infty} A_n \in \mathcal{A}$
    \end{enumerate}
\end{definition}

Observe the following:
\begin{itemize}
    \item $\mathbb{R} \in \mathcal{A}$ always
    \item If $\{ A_n \}_{n=1}^{N} \subset \mathcal{A}$ then $\cup_{n=1}^{N} A_n \in \mathcal{A}$ (just define $A_n = \emptyset$ for $n > N$)
    \item If $\{ A_n \}_{n=1}^{\infty} \subset \mathcal{A}$ then $\cap_{n=1}^{\infty} A_n \in \mathcal{A}$ (since $(\cap_{n=1}^{\infty} A_n)^c = \cup_{n=1}^{\infty} A_{n}^{c}$)
    \item If $A,B \in \mathcal{A}$ then $A \setminus B \in \mathcal{A}$ too since $A \setminus B = A \cap B^c$
\end{itemize}

\underline{\textbf{Examples}:}

\begin{enumerate}
    \item $\mathcal{A} = \{ \emptyset, \mathbb{R} \}$ ``Minimal $\siga$''
    \item $\mathcal{A} = \mathcal{P}(\mathbb{R}) = $ Collection of all subsets of $\mathbb{R}$.
    ``Maximum $\siga$''
\end{enumerate}

In fact, if $\mathcal{A}$ is any $\siga$, then $\{ \emptyset,\mathbb{R} \} \subseteq \mathcal{A} \subseteq \mathcal{P}(\mathbb{R})$

For better examples, let $F$ be any collection of subsets of $\mathbb{R}$.
I want to make $F$ into a $\siga$.
Define $m = \{ \mathcal{A} \mid \mathcal{A} \text{ is a } \siga \text{ that satisfies } F \subset \mathcal{A} \}$.
$m \neq \emptyset$ since it contains $\mathcal{P}(\mathbb{R})$

If $\mathcal{A}, \mathcal{B} \in m$, I can define $\mathcal{A} \cap \mathcal{B} = \{ A \subset \mathbb{R} \mid A \in \mathcal{A} \text{ and } A \in \mathcal{B} \}$ and I can do the same for $\cap_{i \in I} \mathcal{A}$ arbitrary intersection of $\siga$ is still a $\siga$

Define $\hat{F_i} = \cap_{\mathcal{A} \in m} \mathcal{A}$ as a $\siga$ and $F \subset \hat{F}$ and it is the minimal $\siga$ with these properties.
If $G$ is a $\siga$ with $F \subset G$, then $\hat{F} \subset G$.
$\hat{F}$ is the $\siga$ generated by $F$.
Concretely, $\hat{F}$ consists of all subsets of $\mathbb{R}$ that can be constructed by applying countable unions, intersections, and complements to elements of $F$.

\begin{definition}[Borel Sets]
    The $\siga \; \mathcal{B}$ of Borel Sets is the $\siga \; \hat{F}$ generated by
    \[
        F = \{ U \subset \mathbb{R} \mid U \text{ open } \}
    \]
\end{definition}

\begin{remark}
    $\mathcal{B}$ is also the $\siga$ generated by the family of all closed subsets of $\mathbb{R}$
\end{remark}

Singletons $\{ x \} \subset \mathbb{R}$ are closed so if $A \subset \mathbb{R}$ is at most countable then $A$ is Borel.
(e.g $\mathbb{Q} \subset \mathbb{R}$) (e.g $\mathbb{R} \setminus \mathbb{Q}$)

Not all Subsets of $\mathbb{R}$ are Borel.
One can actually show that the cardinality of $\mathcal{B}$ is the same as the cardinality of $\mathbb{R}$.
On the other hand $\mathcal{P}(\mathbb{R})$ has strictly larger cardinality.


    \section{Lebesgue Outer Measure}\label{sec:lebesgue-outer-measure}
    We are hoping to measure the size of subsets of $\mathbb{R}$.
Ideally we would like to find or construct a function
\[
    m: \pwr{R} \rightarrow \mathbb{R}_{\geq 0} \cup \{ + \infty \} = [0, \infty]
\]
Which satisfies the following measure requirements:
\begin{enumerate}
    \item If $I=[a,b]$ or $(a,b)$ or $[a,b)$, or $(a,b]$, $a,b \in \mathbb{R}, a \leq b$ then $m(I) = b-a = $ measure of interval
    \item $m$ is translation invariant.
    i.e if $E \subset \mathbb{R}$ and $x \in \mathbb{R}$, let $E + x = \{ y+x \mid y \in E \}$ then $m(E+x)$ = m(E)
    \item If $\{ E_j \}_{j=1}^{n}$ is a finite collection of pairwise disjoint $E_j \subset \mathbb{R}$ then
    \[
        m \left( \cup_{j=1}^{n} E_j \right) = \sum_{j=1}^{n} m(E_j)
    \]
    \item The same as (3) except for $n = \infty$
\end{enumerate}

\begin{theorem}
    There is no such $m$ satisfying all 4 requirements
\end{theorem}

The proof for this will come later.
The solution for this is that we do not try to measure all subsets of $\mathbb{R}$.
So we have $m: \pwr{R} \rightarrow [0, \infty]$ but now we will just be happy with $m: \mathcal{A} \rightarrow [0,\infty]$ where $\mathcal{A}$ is a $\siga$ which has enough elements.
For example $\mathcal{A} > \mathcal{B}$.

We will follow H. Lebesgue as we proceed in two steps.

\underline{Step 1:} construct Lebesgue outer measure $m^{\star}: \pwr{R} \rightarrow [0, \infty]$ satisfying requirements 1,2, and 3.

\underline{Step 2:} Use $m^{\star}$ to define $\mathcal{A}$ and let $m \subset m^{\star} \mid \mathcal{A}$

To create this Lebesgue outer measure on $\mathbb{R}$ we satisfy a weakened version of requirement (3) that can be called (3w).
For any countably infinite collection $\{ E_j \}_{j=1}^{\infty}$ of arbitrary subsets $E_j \subset \mathbb{R}$
\[
    m^{\star}(\cup_{j=1}^{\infty} E_j) \leq \sum_{j=1}^{\infty} m(E_j)
\]
\begin{theorem}[Lebesgue Outer Measure]
    There is a map $m^{\star}: \pwr{R} \rightarrow \rinf $ that satisfies the measure requirements 1, 2, and 3w.
\end{theorem}

This $m^{\star}$ is called the Lebesgue outer measure on $\mathbb{R}$.

How do we define outer measure $m^{\star}(A)$?

Observe that any $A \subseteq \mathbb{R}$ can be covered by some countable infinite collection $\{ I_j \}_{j=1}^{\infty}$ of bounded open intervals, which are allowed to be empty, but we do not assume that $I_j$ be pairwise disjoint.

For example: $I_j = (-j, j), \; j=1,2,3\hdots$

Let
\[
    \mathcal{C}_A = \{ \{I_j\}_{j=1}^{\infty} \mid I_j \text{ bounded open intervals such that } A \subset \cup_{j=1}^{\infty} I_j \}
\]
$\mathcal{C}_A \neq \emptyset$ by our example so for each $\{ I_j \} \in \mathcal{C}_A$, I can consider
\[
    \sum_{j=1}^{\infty} \ell(I_j) \in \rinf \tag{$\ell$ denotes length}
\]

\begin{definition}[Outer Measure]
    \[
        \boxed{m^{\star}(A) \coloneqq \inf_{\{ I_j \} \in \mathcal{C_A}} \sum_{j=1}^{\infty} \ell (I_j)  } \in \rinf
    \]
\end{definition}

This defines a map $m^{\star}: \pwr{R} \rightarrow \rinf$

\underline{Simple Properties:}

\begin{itemize}
    \item \emph{Monotonicity}:
    If $A \subseteq B$ then $m^{\star}(A) \leq m^{\star}(B)$.
    Indeed by definition $\mathcal{C}_B \subseteq \mathcal{C}_A$ hence the infimum over $\mathcal{C}_B$ is $\geq$ than the infimum over $\mathcal{C}_A$.
    \item \emph{Empty Set}: $\mstar{\emptyset} = 0$.
    Given any $1 > \epsilon > 0$, let $I_j = (-\epsilon^{j}, \epsilon^j), \; j=1,2,\hdots$
    $\{I_j\} \in \mathcal{C}_{\emptyset}$ and $\sum_{j=1}^{\infty} \ell (I_j) = 2 \sum_{j=1}^{\infty} \epsilon^j = \frac{2\epsilon}{1-\epsilon} $ from the geometric series going to zero so $\mstar{\emptyset} \leq \frac{2\epsilon}{1-\epsilon} \; \forall 0 < \epsilon < 1$
    \item If $A \in \mathbb{R}$ is finite or countable infinite then $\mstar{A} = 0$.
    Indeed enumerate all elements of $A$ by $\{ a_j \}_{j=1}^{\infty}$.
    (If $A$ is finite say $\card{A} = n$ let $a_j = a_n$ for all $j > n$).
    For any $0 < \epsilon < 1$, let $I_j = \left( -\epsilon^j + a_j, a_j + \epsilon^j \right)$ so $A \subseteq \cup_{j=1}^{\infty} I_j$ and $\isum{j} \ell (I_j) = \frac{2\epsilon}{1-\epsilon}$ hence as before, $\mstar{A} = 0$.
    For example $\mstar{\mathbb{Q}} = 0$
\end{itemize}

We will now prove that the Lebesgue outer measure satisfies 1, 2, and 3w of the measure requirements.

\underline{Proof of Property 1}:
i.e $\mstar{I} = \ell(I)$ for any interval $I \subseteq \mathbb{R}$

Assume that $I = [a,b]$, $a < b$ are finite numbers.
Assume that $I$ is a bounded closed interval.
Our goal is to show that $\mstar{I} = b -a$.
One direction of inequality is easy to prove, the other is quite tedious and will be left out.

For any $\epsilon > 0$ let $I_1 = (a-\epsilon, b + \epsilon) > I$, let $I_j = \emptyset, j \geq 2$ so $\{I_j \} \in \mathcal{C}_I \Rightarrow \mstar{I} \leq \isum{j} \ell (I_j) = b - a + 2 \epsilon$.
Let $\epsilon \rightarrow 0$ and we obtain $\mstar{I} \leq b-a$.

\underline{Proof of Property 2}:
i.e $\forall A \subset \mathbb{R}, \forall x \in \mathbb{R}$, $\mstar{A+x} = \mstar{A}$

$\mathcal{C}_A$ and $\mathcal{C}_{A+x}$ are naturally in bijection via $\{ I_j \} \leftrightarrow \{ I_j + x \}$.
Furthermore $\ell (I_j + x) = \ell (I_j)$
\begin{align*}
    \mstar{A+x} &= \inf_{\{I_j +x\} \in \mathcal{C}_{A+x}} \isum{j} \ell (I_j + x) \\
    &= \inf_{\{I_j\} \in \mathcal{C}_{A}} \isum{j} \ell (I_j) = \mstar{A}
\end{align*}

\underline{Proof of Property 3w}:
i.e If $\{ E_j \}_{j=1}^{n}$ is a finite collection of pairwise disjoint $E_j \subset \mathbb{R}$ then $\mstar{\cup_{j=1}^{n} E_j} \leq \sum_{j=1}^{n} \mstar{E_j}$

If $\mstar{E_j} = +\infty$ for some $j$, then the property holds.
We may assume that $\mstar{E_j} < + \infty \; \forall j$.
Let $\epsilon > 0$.
By the definition of infimum, for each $j \geq 0$, there is
\[
    \{ I_{j,k} \}_{k=1}^{\infty} \in \mathcal{C}_{E_j} \text{ such that } \isum{k} \ell (I_{j,k}) < \mstar{E_j} + \epsilon 2^{-j}
\]
Thus $\{ I_{j,k} \}_{k=1}^{\infty}$ is still countable and it covers $\cup_{j=1}^{\infty} E_j$ meaning it belongs to $\mathcal{C}_{\cup_{j=1}^{\infty}} E_j$, so by definition
\[
   \mstar{\bigcup_{j=1}^{\infty} E_j} \leq \sum_{j=1}^{\infty} \isum{k} \ell (I_{j,k}) < \sum_{j=1}^{\infty} (\mstar{E_j} + \epsilon 2^{-j}) = \sum_{j=1}^{\infty} \mstar{E_j} + \epsilon
\]
Then let $\epsilon \rightarrow 0$.
Clearly, by taking all $E_j = \emptyset$ except finitely many, we have the same subadditivity 3w for finite collections.

\begin{corollary}
    $\mstar{[0,1] \cap (\mathbb{R} \setminus \mathbb{Q})} = 1 = \ell([0,1])$
\end{corollary}
\begin{proof}
    \begin{align*}
        \mstar{[0,1] \cap (\mathbb{R} \setminus \mathbb{Q})} &\leq \mstar{[0,1]} = 1 \\
        &\leq \mstar{[0,1] \cap (\mathbb{Q})} + \mstar{[0,1] \cap (\mathbb{R} \setminus \mathbb{Q})} \\
        &\leq 0 + 1
    \end{align*}
\end{proof}

\begin{corollary}
    $\mathbb{R} \setminus \mathbb{Q}$ is uncountable
\end{corollary}
\begin{proof}
    If not, then
    \[
        \mstar{\mathbb{R} \setminus \mathbb{Q}} = 0 \geq \mstar{[0,1] \cap (\mathbb{R} \setminus \mathbb{Q})} = 1 \qedhere
    \]
\end{proof}


    \section{The $\sigma$-Algebra Of Lebesgue Measurable Sets}\label{sec:the-$sigma$-algebra-of-lebesgue-measurable-sets}
    $m^{\star}$ does not satisfy the third measurability requirement without the weak 3w condition.
We can construct some examples to prove this.
$A, B \subset \mathbb{R}, A \cap B = \emptyset$, such that $\mstar{A \cup B} < \mstar{A} + \mstar{B}$ later in the class.

The idea to avoid this problem is to look at ``reasonable'' subsets of $\mathbb{R}$ for which this paradox disappears.

\begin{definition}[Carath\'eodory]
    $E \subseteq R$ is called (Lebesgue) measurable if $\forall A \subset \mathbb{R}$
    \[
        \mstar{A} = \mstar{A \cap E} + \mstar{A \cap E^c}
    \]
\end{definition}
\begin{remark}
    This is equivalent to Lebesgue's definition: $E$ is measurable if and only if
    \[
        \exists U \subset \mathbb{R} \text{ such that } E \subset U \text{ and } \mstar{U \setminus E} < \epsilon
    \]
    But we will discuss this later.
\end{remark}

Suppose that $A$ is measurable and $B \subset \mathbb{R}$ is any set such that $A \cap B = \emptyset$ then
\[
    \mstar{A \cup B} = \mstar{\underbrace{(A \cup B) \cap A}_{=A}} + \mstar{\underbrace{(A \cup B) \cap A^{c}}_{=B}}
\]
Going back to our counter example for $m^{\star}$ and measurability requirement 3, $A$ or $B$ would have to be unmeasurable.

Here's another observation: For $E, A \subset \mathbb{R}$ arbitrary sets we have
\[
    A = (A \cap E) \cup (A \cap E^{c})
\]
So by 3w $\mstar{A} \leq \mstar{A \cap E} + \mstar{A \cap E^{c}}$, so $E$ is measurable $\iff$ $\forall A \subset \mathbb{R}$
\[
    \boxed{\mstar{A} \geq \mstar{A \cap E} + \mstar{A \cap E^{c}}}
\]
This holds trivially for $\mstar{A} = \infty$

\underline{Example 1:} $\emptyset$ is measurable.
$\forall A \subset \mathbb{R}$
\[
    \mstar{A} = \cancel{\mstar{A \cap \emptyset}} + \mstar{A \cap \mathbb{R}}
\]

\underline{Example 2:} $\mathbb{R}$ is measurable.
$\forall A \subset \mathbb{R}$
\[
    \mstar{A} = \mstar{A \cap \mathbb{R}} + \mstar{A \cap E^{c}}
\]

\begin{prop}
    $E \subset \mathbb{R}$ with $\mstar{E} = 0$, then $E$ is measurable.
\end{prop}

\begin{*corollary}
    Every countable set is measurable.
    $\mathbb{Q}$ measurable $\rightarrow \mathbb{R} \setminus \mathbb{Q}$ are measurable
\end{*corollary}

\begin{proof}
    Let $A \subset \mathbb{R}$ be any set
    \begin{align*}
        A \cap E \subset E &\Rightarrow \mstar{A \cap E} \leq \mstar{E} = 0 \\
        A \cap E^{c} \subset A &\Rightarrow \mstar{A \cap E^{c}} \leq \mstar{A} \\
        \text{ So } \mstar{A} &\geq \mstar{A \cap E^c} + \cancel{\mstar{A \cap E}}
    \end{align*}
\end{proof}

Our goal is to show that Lebesgue measurable sets $\mathcal{L} = \{ E \subset \mathbb{R} \mid E \text{ is measurable} \}$ is a $\siga$ on $\mathbb{R}$.
We just need to show that if $\{ E_j \}_{j=1}^{\infty}$ with $E_j \in \mathcal{L}, \; \forall j$, then $\cup_{j=1}^{\infty} E_j \in \mathcal{L}$

\begin{prop}
    If $\{ E_j \}_{j=1}^{n} \subset \mathcal{L}$ then $\cup_{j=1}^{n} E_i \in \mathcal{L}$
\end{prop}

\begin{proof}
    We use mathematical induction.
    $n=1$ is trivial so we set the base case as $n=2$.
    $E_1, E_2$ are measurable, Let $A \subset \mathbb{R}$ be any set
    \begin{align*}
        \mstar{A} &= \mstar{E_1 \cap A} + \mstar{A \cap E_1 ^{c}} \\
        &= \mstar{A \cap E_1} + \mstar{(A \cap E_1^{c}) \cap E_2} + \mstar{(A \cap E_1 ^{c}) \cap E_2 ^{c}} \\
        &= \mstar{A \cap E_1} + \mstar{(A \cap E_1^{c}) \cap E_2} + \mstar{A \cap (E_1 ^{c} \cap E_2 ^{c})} \\
        &= \mstar{A \cap E_1} + \mstar{(A \cap E_1^{c}) \cap E_2} + \mstar{A \cap (E_1 \cup E_2)^{c}}\\
        &\geq \mstar{A \cap (E_1 \cup E_2)} + \mstar{A \cap (E_1 \cup E_2)^{c}} \tag{3w} \\
    \end{align*}
    So $E_1 \cup E_2 \in \mathcal{L}$.

    Induction step $n \geq 2$
    \[
        \bigcup_{j=1}^{\infty} E_j = \left( \bigcup_{j=1}^{n-1} E_j \right) \cup E_n \in \mathcal{L} \text{ by the $n=2$ case} \qedhere
    \]
\end{proof}

To prove that this also applies to countable sets, we use
\begin{prop}[Analog of measurability requirement 3 for $m^{\star} \mid \mathcal{L}$]
    Suppose $A \subset \mathbb{R}$ is any set and $\{ E_j \}_{j=1}^{n}$ is a finite disjoint collection of sets $E_j \in \mathcal{L}$, then
\[
    \mstar{A \cap \bigcup_{j=1}^{n} E_j} = \sum_{j=1}^{n} \mstar{A \cap E_j}
\]
    In particular take $A = \mathbb{R}$ to get $\mstar{\bigcup_{j=1}^{n}E_j} = \sum \mstar{E_j}$
\end{prop}

\begin{prop}
    If $\{ E_j \}_{j=1}^{\infty}$ is a countable family with $E_i \in \mathcal{L} \; \forall j$, then $\cup_{j=1}^{\infty} E_j \in \mathcal{L}$.
    In particular, $\mathcal{L}$ is a $\siga$.
\end{prop}

We would like to have the Borel sets be measurable, i.e  $\mathcal{B} \subset \mathcal{L}$.
Recall that $\mathcal{B} = \hat{\mathcal{F}}$, where $\mathcal{F} = \{ U \subset \mathbb{R} \mid U \text{ is open } \}$ and \verb!^! denotes the $\siga$.

This results follows from the measurability of intervals combined with the measurability of the union of measurable sets.

\begin{prop}
    If $I \subseteq \mathbb{R}$ is any interval, then $I$ is measurable.
\end{prop}

\begin{theorem}
    $\mathcal{L} =$ Lebesgue Measurable subsets of $\mathbb{R}$ form a $\siga$ that contains the Borel $\siga$ $\mathcal{B}$
\end{theorem}

\begin{proof}
    We already know that $\mathcal{L}$ is a $\siga$.
    If we can show that $\mathcal{L}$ contains all open sets $U \subset \mathbb{R}$, then $\mathcal{L}$ (being a $\siga$) must contain $\mathcal{B}$ which is the $\siga$ generated by open sets.
    Now if $U \subset \mathbb{R}$ is any (non empty) open set then by definition $\forall x \in U, \exists I_x \ni x $ where $I_x$ is an open interval and $I_x \subset U$.

    We want to choose $I_x$ to be the ``maximal'' such.
    So by assigning
    \[
        a_x \coloneqq \inf \{z \in \mathbb{R} \mid (z,x) \subset U \} \text{ satisfies } a_x < x
    \]
    and
    \[
        b_x \coloneqq \sup \{y \in \mathbb{R} \mid (x,y) \subset U \} \text{ satisfies } x < b_x
    \]
    so $I_x \coloneqq (a_x,b_x)$ is an open interval that contains $x$ and by construction $I_x \in U$.
    It is the largest such, in the sense that if $a_x > - \infty$ then $a_x \notin U$ and symmetrically if $b_x < \infty$ then $b_x \notin U$.

    For any $y \in I_x$, we have $y<b_x$, so there is $z > y$ such that $(x,z) \subset U$ so $y \in U$.
    Indeed, if $a_x \in U$ then since $U$ open, $\exists r > 0$ such that $(a_x - r, a_x +r) \subset U$ contradicting the definition of $a_x$.

    So $U = \cup_{x \in U} I_x$.
    It is a huge union, however if $x, x\pr \in U, x \neq x\pr$, then either $I_x \cap I_{x\pr} = \emptyset$, or if not then necessarily $I_x = I_{x\pr}$, since $I_x \cup I_{x\pr}$ is then another open interval that contains $x$ \& $x\pr$ and is a subset of $U$, so by maximality it must equal $I_x$ \& $I_{x\pr}$.
    So, throwing away all repeated $I_x$, we can write $U = \cup_{i \in I} I_x$ for some $I$ where the intervals $I_{x_i}$ are pairwise disjoint.
    By density of $\mathbb{Q} \subset \mathbb{R}$, each such interval contains a different rational number $r_i \in I_{x_i}$.
    Since $\mathbb{Q}$ is countable, $I$ is at worst countable.

    So every $U$ open is an at most countable disjoint union of open intervals.
    Since such intervals belong ot $\mathcal{L}$, and $\mathcal{L}$ is a $\siga$, it follows that every $U$ open is in $\mathcal{L}$ as desired.
\end{proof}

\begin{prop}[The $\siga \; \mathcal{L}$ is also translation invariant]
    If $E \subset \mathcal{L}$ and $x \in \mathbb{R}$ then $E + x \in \mathcal{L}$
\end{prop}

\begin{proof}
    Given any $A \subset \mathbb{R}$,
    \begin{align*}
        \mstar{A} &= \mstar{A - x} \\
        &= \mstar{(A - x) \cap E} + \mstar{(A-x) \cap E^c}  \\
        &= \mstar{A \cap E + x} + \mstar{A \cap (E+x)^c} \tag{$m^{\star}$ translation invariant }
    \end{align*}
\end{proof}

\begin{remark}
    If $A \in \mathcal{L}$ with $\mstar{A} < \infty$, and $B \subset \mathbb{R}$ is any set with $A \subset B$, then
    \[
        \mstar{B \setminus A} = \mstar{B} - \mstar{A}
    \]
\end{remark}


    \section{Outer and Inner Approximation of Lebesgue Measurable Sets}\label{sec:outer-and-inner-approximation-of-lebesgue-measureable-sets}
     \begin{definition}[Gebiet-Durchshnitt]
     A subset $A \subset \mathbb{R}$ is called a $G_{\delta}$ if $A = \cap_{i=1}^{\infty} A_i$ where $A_i$ are all open (possibly empty).
 \end{definition}

 \begin{definition}[Ferm\'e-Somme]
     A subset $A \subset \mathbb{R}$ is called a $F_\sigma$ if $A = \cup_{i=1}^{\infty} A_i$ where $A_i$ are all closed (possibly empty).
 \end{definition}

 Clearly, $A$ is $G_\delta \iff A^c$ is $F_\delta$.
 Also clearly, all $\gdelta$ and $\fsigma$ sets are Borel.
 Of course not all $\gdelta$ are open, e.g $[0,1] = \cap_{i=1}^{\infty} \left(- \frac{1}{i}, 1 + \frac{1}{i}\right)$ and not all $\fsigma$ are closed.
 e.g. $(0,1) = \cup_{i=1}^{\infty} \left[\frac{1}{i}, 1-\frac{1}{i}\right]$

$\mathbb{Q}$ is clearly $\fsigma$, so $\mathbb{R} \setminus \mathbb{Q}$ is $\gdelta$.
 With this, we can give several equivalent formulations of measurability.

 \begin{theorem}
     Let $E \subset \mathbb{R}$ be any set, then the following are equivalent:
     \begin{enumerate}
         \item $E \in \mathcal{L}$
         \item $\forall \epsilon > 0$, $\exists U \supset E$, $U$ open, $\mstar{U \setminus E} < \epsilon$
         \item $\exists G \subset \mathbb{R}$ a $\gdelta$ set, $G \supset E$, with $\mstar{G \setminus E} = 0$
         \item $\forall \epsilon > 0$, $\exists F \subset E$, $F$ closed, $\mstar{E \setminus F} < \epsilon$
         \item $\exists F \subset \mathbb{R}$ a $\fsigma$ set, $F \subset E$ with $\mstar{E \setminus F} = 0$
     \end{enumerate}
 \end{theorem}

 \begin{prop}
     For an $E \in \mathcal{L}$ with $\mstar{E} < \infty$.
     Then $\forall \epsilon > 0$, $\exists \{ I_j \}_{j=1}^{n}$ a finite disjoint family of open intervals so that if we let $U = \cup_{j=1}^{n} I_j$ (open) then $\mstar{E \Delta U} < \epsilon$.
 \end{prop}


    \section{Lebesgue Measure}\label{sec:lebesgue-measure}
    We can now take $m^{\star}$ and restrict it to $\mathcal{L}$. $m^{\star} \mid_{\mathcal{L}}$.
\begin{definition}[Lebesgue Measure]
    This Lebesgue Measure is a function
    \[
        m \coloneqq m^{\star} \mid_{\mathcal{L}} : \mathcal{L} \rightarrow \rinf
    \]
\end{definition}
This means that for $E \in \mathcal{L}$ we define $m(E) = \mstar{E}$.
Clearly, $m$ satisfies the measurability requirements 1, 2, \& 3 as we have proved earlier.
It also satisfies requirement 4 which was requirement 3 for countably infinite sets.

\begin{prop}
    If $\{ E_j \}_{j=1}^{\infty}$ is a countably infinite collection of pairwise disjoint sets $E_j \in \mathcal{L}$ (possibly empty), then $\cup_{j=1}^{\infty} E_j \in \mathcal{L}$ and
    \[
        m\left(\bigcup_{j=1}^{\infty} E_j\right) = \sum_{j=1}^{\infty} m(E_j)
    \]
\end{prop}

\begin{proof}
    We proved earlier that $\cup_{j=1}^{\infty} E_j \in \mathcal{L}$ and that
    \[
        m\left(\bigcup_{j=1}^{\infty} E_j\right) \leq \sum_{j=1}^{\infty} m(E_j)
    \]
    For the opposite inequality, for each $n$ we proved earlier that
    \[
        m\left(\bigcup_{j=1}^{n} E_j\right) = \sum_{j=1}^{n} m(E_j)
    \]
    But $\cup_{j=1}^{n} E_j \subset \cup_{j=1}^{\infty} E_j$, hence
    \[
        m\left(\bigcup_{j=1}^{\infty} E_j\right) \geq m\left(\bigcup_{j=1}^{n} E_j\right) =  \sum_{j=1}^{n} m(E_j) \;\; \forall n
    \]
    Take the limit as $n \rightarrow \infty$ to get
    \[
        m\left(\bigcup_{j=1}^{\infty} E_j\right) \geq \sum_{j=1}^{\infty} m(E_j)
    \]
    As desired.
    This argument shows that measurability requirement 3 and 3w together imply 4.
\end{proof}


    \section{Non-Measurable Sets}\label{sec:non-measurable-sets}
    We saw earlier that if $E \subset \mathbb{R}$ satisfies $\mstar{E} = 0$ then $E \in \mathcal{L}$.
In particular, $\forall F \subset E$, $\mstar{F} \leq \mstar{E} = 0$, so $F \in \mathcal{L}$ too.
This however totally fails when $\mstar{E} > 0$.

\begin{theorem}[Vitali]
    For any $E \subset \mathbb{R}$ with $\mstar{E} > 0$, there is an $F \subset E$ which is NOT measurable.
    The construction uses the axiom of choice (and it is really needed).
\end{theorem}

The proof of this theorem and construction of a Vitali set are currently omitted due to length.


    \section{Cantor Set}\label{sec:cantor-set}
    We showed earlier that if $A \subset \mathbb{R}$ is countable then $A \in \mathcal{L}$ and $m(A) = 0$.
How about the converse;
if $A \in \mathcal{L}$ has $m(A) = 0$, is $A$ countable?
No!

\begin{theorem}[Cantor]
    There is a closed, uncountable set $\mathcal{C}$ with $m(\mathcal{C}) = 0$
\end{theorem}

Start with an interval $I = [0,1]$ and remove the middle $\frac{1}{3}$, namely $(\frac{1}{3}, \frac{2}{3})$.
\begin{align*}
    \mathcal{C}_1 &\coloneqq I \setminus \left(  \frac{1}{3}, \frac{2}{3} \right)  = \left[ 0, \frac{1}{3} \right] \bigcup \left[ \frac{2}{3}, 1 \right] \\
    \mathcal{C}_2 &\coloneqq  \mathcal{C}_1 \setminus \left( \left( \frac{1}{9}, \frac{2}{9} \right) \bigcup  \left( \frac{7}{9}, \frac{8}{9} \right) \right) \\
    \mathcal{C}_k &\coloneqq  \mathcal{C}_{k-1} \setminus \bigcup_{j=0}^{3^{k-1}-1} \left( \frac{3j + 1}{3^k} , \frac{3j + 2}{3^k}\right) \\
    &= [0,1] \setminus \bigcup_{l=1}^{k} \bigcup_{j=0}^{3^{l-1}-1} \left( \frac{3j + 1}{3^l} , \frac{3j + 2}{3^l}\right)
\end{align*}

Thus $\{ \mathcal{C}_k \}_{k=1}^{\infty}$ is a very large descending (i.e nested $\mathcal{C} \subset \mathcal{C}_{k-1}$) sequence of closed sets, and $\mathcal{C}_k$ is a disjoint union of $2^k$ closed intervals of length $\frac{1}{3^k}$.
Let then $\mathcal{C} = \bigcap_{k=1}^{\infty} \mathcal{C}_k$, so $\mathcal{C}$ is closed, and hence also measurable.

Since $m(\mathcal{C}_k) = \left( \frac{2}{3} \right)^k$, $m(\mathcal{C}) \leq m(\mathcal{C}_k) \leq \left( \frac{2}{3} \right)^k \; \forall k$.
Taking the limit as $k \rightarrow \infty$ we get $m(\mathcal{C}) = 0$.

Suppose that $\mathcal{C}$ was countable, let $\{ c_k \}_{k=1}^{\infty}$ be an enumeration of all it's elements.
Then writing $\mathcal{C}_1 = $ the disjoint union of 2 interavals, we must have that $c_1$ belongs to precisely one of them.
Say $c_1 \notin F_1$.
Now $F_1 \subset \mathcal{C}_2$ is made of 2 disjoint intervals, and one of them does not contain $c_2$, say $c_2 \notin F_2$.

Continue this way until we get a sequence of $\{ F_k \} _{k=1}^{\infty}$, where $F_k$ is a closed interval, $F_{k+1} \subset F_k$, and $F_k \subset \mathcal{C}_k$, and $c_k \notin F_k$.
By the nested set theorem, let $x \in \cap _{k=1}^{\infty} F_k$.
Then
\[
    x \in \cap _{k=1}^{\infty} F_k \subset \cap _{k=1}^{\infty} \mathcal{C}_k = \mathcal{C}
\]
So $x \in \mathcal{C}$ but $\{ c_k \}_{k=1}^{\infty}$ enumerates ALL points of $\mathcal{C}$ so $\exists n$ such that $x = c_n$.
Hence $x \notin F_n$ but this is a contradiction so we conclude that $\mathcal{C}$ is uncountable.

Finally observe that $\mathcal{C}$ is closed and $\mathcal{C} \subset [0,1]$, so $\mathcal{C}$ is compact by Heine-Borel.

There are two variations of this theorem.
\begin{enumerate}
    \item If instead of removing the middle third, we removed the middle $p\%$ where $0 < p < 100$, then we also get a Cantor set which has the same properties as $\mathcal{C}$.
    \item We could also remove a \emph{smaller} proportion at each step, instead of a fixed one.
    At each step we remove $2^{n-1}$ intervals of length $a^n$ for some $0 < a \leq \frac{1}{3}$.
    Then the total length removed is $\sum_{n=1}^{\infty} 2^{n-1} a^n = \frac{a}{1-2a}$.
    So, for this ``fat'' Cantor set $m(\mathcal{C}_{\text{fat}}) = 1 - \frac{a}{1-2a} = \frac{1 - 3a}{1-2a}$.
    Which is indeed 0 when $a = \frac{1}{3}$ (standard Cantor), and $m(\mathcal{C}_{\text{fat}}) > 0$ for $0 < a < \frac{1}{3}$
\end{enumerate}

\begin{remark}
    $\card{\mathcal{L}} = \card{\pwr{R}}$: $\leq$ is trivial so $\forall A \subset \mathcal{C}$, $A \in \mathcal{L}$ but $\card{\mathcal{C}} = \mathbb{R} \Rightarrow \card{\mathcal{L}} = \card{\pwr{R}}$
\end{remark}

\begin{remark}
    $\card{\pwr{R} \setminus \mathcal{L}} = \card{\pwr{R}}$: Let $V$ be a Vitali set, $V \subest [0,1]$, then $\forall A \subset [2,3], V \cup A \notin \mathcal{L}$ and so $\card{\pwr{R}} \geq \card{\pwr{R} \setminus \mathcal{L}} \geq \card{\mathcal{P}([2,3])} = \card{\pwr{R}} $
\end{remark}

\textbf{\underline{Cantor-Lebesgue Function}}

Let $U_k \coloneqq [0,1] \setminus \mathcal{C}_k$, which is $2^k - 1$ disjoint open intervals, of various lengths, and
\[
    U = [0,1] \setminus \mathcal{C} = [0,1] \setminus \bigcap_{k=1}^{\infty} \mathcal{C}_k = \bigcup_{k=1}^{\infty} U_k
\]
Thus $U$ is open on $[0,1]$ and $m(U) = m([0,1]) = 1$ since $m(\mathcal{C}) = 0$.

\begin{theorem}
    There is a continuous (weakly) increasing function $\phi : [0,1] \rightarrow [0,1]$ that is surjective with $\phi(0) = 0$ and $\phi(1) = 1$ such that $\phi$ is differentiable in $U$ and $\phi \pr (x) = 0$ $\forall x \in U$
\end{theorem}

First define $\phi$ on $U_k$ by setting it to be equal to the constants $\{ \frac{1}{2^k}, \frac{2}{2^k}, \hdots, \frac{2^k - 1}{2^k} \}$ on it's $2^k -1$ open intervals.
Observe that if we increase $k \rightarrow k+1$, $U_{k+1}$ has more intervals but some of them are the same that we already had in $U_k$, and on those, the value of $\phi$ in the 2 steps agrees!

Taking the union over $k$ defines $\phi$ on $U$.
To extend $\phi$ to all of $[0,1]$, we let $\phi(0) = 0$ and for all $x \in \mathcal{C} \setminus \{ 0 \}$ let $\phi \card{x} \coloneqq \sup \{ \phi(y) \mid y \in U \cap [0,x) \}$ (this is finite since $ \leq 1$)

We have defined a function $\phi: [0,1] \rightarrow [0,1)$ and it satisfies the specified properties.

Consider now $\psi (x) \coloneqq \phi(x) + x$ for $x \in [0,1]$.
Some obvious properties:
\begin{itemize}
    \item $\psi$ is continuous
    \item $\psi$ is strictly increasing
    \item $\psi(0) = 0$, $\psi(1) = 2$
    \item $\psi([0,1]) = [0,2]$ and $\psi$ is a bijection between these
    \item $\psi^{-1}: [0,2] \rightarrow [0,1]$ is continuous
\end{itemize}

\begin{prop}
    $m(\psi(\mathcal{C})) = 1$ and $\exists E \subset \mathcal{C}$, $E \in \mathcal{L}$ such that $\psi(E) \notin \mathcal{L}$
\end{prop}

\begin{*corollary}
    This set $E$ is measurable but not Borel.
\end{*corollary}

\begin{prop}[Continuity of Measure]

    \begin{enumerate}
        \item If $\{ A_j \}_{j=1}^{\infty}$ are measurable sets with $A_j \subset A_{j+1} \; \forall j$, then
        \[
            m\left( \bigcup_{j=1}^{\infty} A_j \right) = \lim_{j \rightarrow \infty} m(A_j)
        \]
        \item  If $\{ B_j \}_{j=1}^{\infty}$ are measurable sets with $B_{j+1} \subset B_j$ $\forall j$, and $m(B_j) < \infty \iff m(B_1) < \infty$ then
        \[
            m \left( \bigcap_{j=1}^{\infty} B_j \right) = \lim_{j \rightarrow \infty} m(B_j)
        \]
    \end{enumerate}

\end{prop}

\begin{definition}[Almost Everywhere]
    We say some property ``$P$'' holds almost everywhere on $E$, or for a.e $x \in E$, if $\exists E_0 \subset E$ with $\mstar{E_0} = 0$ such that $P$ holds for \emph{all} $x \in E \setminus E_0$.
    We also say ``$P$ holds for almost all $x$ in $E$''.
\end{definition}

\underline{Ex:} Almost every real number is irrational.

\begin{prop}[Borel-Cantelli's Lemma]
    Let $\{ E_j\}_{j=1}^{\infty} \subset \mathcal{L}$ be such that $\sum_{j=1}^{\infty} m(E_j) < \infty$.
    Then almost every $x \in \mathbb{R}$ belongs to at most finitely many $E_j$'s.
\end{prop}
\begin{proof}
    For each $n$,
    \[
        m\left( \bigcup_{j=n}^{\infty} \right) \leq \sum_{j=n}^{\infty} m(E_j) < \infty
    \]
    and
    \[
        \bigcup_{j=n+1}^{\infty} E_j \subset \bigcup_{j=n}^{\infty} E_j
    \]
    So by the continuity of measure
    \[
        m \left( \bigcap_{n=1}^{\infty} \bigcup_{j=n}^{\infty} E_j \right) = \lim_{n \rightarrow \infty} m \left( \bigcup_{j=n}^{\infty} E_j \right) \leq \lim_{n \rightarrow \infty} \sum_{j=n}^{\infty} m(E_j) \underbrace{=}_{\text{tails of a convergent series}} 0
    \]
    Hence ``almost every'' $x \in E$ satisfies $x \notin \cap_{n=1}^{\infty} \cup_{j=n}^{\infty} E_j$.
    i.e for each such $x$, $\exists n$ such that $x \notin \cup_{j=n}^{\infty} E_n$ so $x$ belongs only to (at most) $E_1 \hdots E_{n-1}$
\end{proof}


    \section{Measurable Functions}\label{sec:measurable-functions}
    We shall now study functions $f : E \rightarrow [-\infty, \infty] \coloneqq \mathbb{R} \cup \{ \pm \infty \}$ where $E \subset \mathbb{R}$ is a \emph{measurable} set.

Sublevel sets of $f$ are the sets of the form $f^{-1}([-\infty, c)) = \{ x \in E \mid f(x) < c \}$, for some $c \in \mathbb{R}$

\begin{definition}
    If we have $f : E \rightarrow [-\infty, \infty]$ with $E$ measurable, then we say that $f$ is measurable if all sublevel sets $f^{-1}([-\infty, c))$ are in $\mathcal{L}$ for all $x \in \mathbb{R}$.
\end{definition}

\begin{prop}
    $f : E \rightarrow [-\infty, \infty]$, then the following are equivalent:
    \begin{enumerate}
        \item $f$ measurable
        \item $\forall c \in \mathbb{R}$, $f^{-1}([-\infty, c]) = \{x \in E \mid f(x) \leq c \} \in \mathcal{L}$
        \item $\forall c \in \mathbb{R}$, $f^{-1}((c, \infty]) = \{x \in E \mid f(x) > c \} \in \mathcal{L}$
        \item $\forall c \in \mathbb{R}$, $f^{-1}([c, \infty])  \in \mathcal{L}$
        \item $\forall U \subset \mathbb{R}$ open, $f^{-1}(U) \in \Le$
        \item $\forall A \subset \mathbb{R}$ Borel set, $f^{-1}(A) \in \Le$
    \end{enumerate}
\end{prop}

\underline{Ex:} If $E$ measurable, $f: E \rightarrow \R$ continuous, then $f$ is measurable.
Indeed, $\forall U \subset \R$ open, $f^{-1}(U)$ is open in $E$, i.e $f^{-1}(U) = V \cap E$ where $V \subset \R$ open.
Clearly $V \cap E \in \Le$, so $f$ is measurable.

\underline{Caution}: $f: E \rightarrow \R$ continuous and $A \subset \R$ measurable $\centernot\implies f^{-1}(A) \in \Le$.
For example: $E = [0,1], f = \psi^{-1}$ then we proved earlier that $\psi$ maps a measurable subset onto a non-measurable subset.

\begin{prop}
    $f: [a,b] \rightarrow \R$ monotone $\implies f$ measurable
\end{prop}

\begin{proof}
    without loss of generality, we may assume $f$ is monotone increasing $f(x) \leq f(y)$ whenever $x \leq y$.
    For any $c \in \R$, look at $\{ f < c \}$ and assume it is non-empty.
    We show that $\{ f < c \}$ is an interval $\subset [a,b]$.
    Now, intervals $I \in \R$ are characterized by the property that if $x \leq y \in I$ then the whole segment $tx + (1-t)y$ is in $I$, for $0 \leq t \leq 1$.
    So let $f(x) < c$, $f(y) < c$, then $tx + (1-t)y \leq y$ so $f(tx + (1-t)y) \leq f(y) < c$ too.

    So $\{ f < c\}$ is an interval which means that $f$ is measurable.
\end{proof}

\begin{prop}
    given $E \subset \R$ measurable, $f: E \rightarrow [-\infty, \infty]$ measurable
    \begin{enumerate}
        \item If $g: E \rightarrow [-\infty, \infty]$ is another function and $f=g$ a.e on $E$.
        Then $g$ is measurable
        \item Suppose $D \subset E$, $D$ measurable.
        Then $f$ is measurable (as a function on $E$) $\iff$ $f\mid_{D}$ measurable (as a function on $D$) and $f \mid_{E \setminus D}$ is measurable (as a function on $E \setminus D$).
    \end{enumerate}
\end{prop}

\begin{proof}
(1): Let $A = \{ x \in E \mid f(x) \ne g(x) \}$, which by assumption has $m(A) = 0$.
    Then $\forall c \in \R$,
    \begin{align*}
        \{ x \in E \mid g(x) > c \} &= \{ x \in A \mid g(x) > c \} \cup \{ x \in E \setminus A \mid f(x) > c \} \\
        &= \{ x \in A \mid g(x) > c \} \cup \underbrace{\{ x \in E\mid f(x) > c \}}_{\in \Le} \cap \underbrace{\{ E \setminus A \}}_{\in \Le}
    \end{align*}
    $\{ x \in A \mid g(x) > c \}$ is a subset of $A$ hence it has measure 0 and is also measurable so $\{ g > c \} \in \Le$.

(2): \begin{align*}
         \{ x \in E \mid f(x) > c \} &= \{ x \in D \mid f(x) > c \} \cup \{ x \in E \setminus D \mid f(x) > c \} \\
         &= ( \{ x \in E \mid f(x) > c \} \cap D) \cup (\{ x \in E \mid f(x) > c \} \cap (E \setminus D))
\end{align*}
\end{proof}

\underline{Sums and Products}: If $f,g : E \rightarrow [-\infty, \infty]$ can we consider their sum $f+g$?
Well, if $f(x) = \infty$ and $g(x) = -\infty$ then $f(x) + g(x)$ is definitely undefined.
Let us then assume that $f$ and $g$ are finite for a.e point in $E$.
Thus, $\exists E_0 \subset E$ with $m(E_0) = 0$, such that $f$ and $g$ are finite on $E \setminus E_0$.
We will now show that $f+g: E \setminus E_0 \rightarrow \R$ is measurable (on $E\setminus E_0$).
Then if $h: E \rightarrow [-\infty, \infty]$ is any function such that $h \mid_{E\setminus E_0} = (f+g)\mid_{E\setminus E_0}$ then $h$ is also measureable by part (2) above.
Observe that such an $h$ always exists (e.g set $h = f+g$ on $E \setminus E_0$ and $h=0$ on $E_0$), and it is not unique at all.
However, as we just said, all such $h$ are measurable.
We thus can say $f+g$ is measurable on $E$.

\begin{prop}
    $f,g: E \rightarrow [-\infty, \infty]$ measurable such that $f,g$ are finite a.e on $E$.
    Then $\forall \alpha, \beta \in \R$, $\alpha f + \beta g$ and $f g$ are measurable on $E$.
\end{prop}

However, composition of two measurable functions may fail to be measurable:

\underline{Ex:} If $E \subset \R$ measurable let $\chi_{E}$ be its characteristic function
\[
    \chi_E (x) = \begin{cases}
                     1 & \text{ if } x \in E \\
                     0 & \text{ if } x \notin E
    \end{cases}
\]
Then $\chi_E$ is measurable on $\R$
\[
    \{ \chi_E < c \} = \begin{cases}
                           \R & c \geq 1 \\
                           E^c & 0 < c < 1 \\
                           \emptyset & c \leq 0
    \end{cases}
\]
Take then $\psi$ from before, $\psi: [0,1] \rightarrow [0,2]$ strictly increasing, with $A \subset [0,1], A \in \mathcal{L}$ and $\psi(A) \notin \Le$.
Extend $\psi$ to $\R$ as strictly increasing and continuous, for example with
\[
    \tilde{\psi} (x) = \begin{cases}
                           \psi(x) & \text{ if } 0 \leq x \leq 1\\
                           x & \text{ if } x < 0 \\
                           2x & \text{ if } x > 1
    \end{cases}
\]
So $\tilde{\psi}: \R \rightarrow \R$ is a strictly increasing continuous bijection which implies $\tilde{\psi}^{-1}: \R \rightarrow \R$ is continuous $\implies \tilde{\psi}^{-1}$ measurable;
$\chi_A$ is also measurable, but $f = \chi_{A} \circ \tilde{\psi}^{-1} : \R \rightarrow \R$ is \emph{NOT} measurable, since if $I = \left( \frac{1}{2}, 2 \right)$, $\chi_{A}^{-1} (I) = A$ then $f^{-1}(I) = \tilde{\psi} \left( \chi_A ^{-1} (I) \right) = \tilde{\psi} (A) = \psi(A) \notin \Le$.

To reconcile this, we introduce the following:
\begin{prop}
    If $g: E \rightarrow \R$ is measurable and $f: \R \rightarrow \R$ continuous then $f \circ g: E \rightarrow \R$ measurable.
\end{prop}

\begin{proof}
    $\forall U \subset \R $ open,
    \[
        (f \circ g)^{-1}(U) = g^{-1}(f^{-1}(U)) \in \Le
    \]
    Since $f^{-1}(U)$ is open and $g$ is measurable.
\end{proof}

\underline{Example:} Take $f: E \rightarrow \R$ measurable, and $p \in \R > 0$.
Then $\card{f}^p: E \rightarrow \R$ is measurable (indeed $y \rightarrow \card{y}^p$ is continuous on $\R$)

\begin{prop}
    Let $\{ f_j \}_{j=1}^{n}$ be a set of measurable functions $E \rightarrow \R$, then $\max_{1 \leq j \leq n} \{ f_j \}$ and $\min_{j} \{ f_j \}$ are measurable.
\end{prop}
\begin{proof}
    \[
        \{ x \in E \mid \max_{j} \{ f_j \} (x) > c \} = \bigcup_{j=1}^{n} \{ x \in E \mid f_j (x) > c \}
    \]
    While $\min \{f_j \} = - \max \{ -f_j \}$
\end{proof}

\textbf{\underline{Convergence of functions:}} $\{ f_n \}_{n=1}^{\infty}$, $f: E \rightarrow [-\infty, \infty], A \subset E$.
We say that $f_n \rightarrow f$ as $n \rightarrow \infty$
\begin{enumerate}
    \item \underline{Pointwise} on $A$ if $\forall x \in A \lim_{n \rightarrow \infty} f_n (x) = f(x)$
    \item \underline{Pointwise a.e} on $A$ if $\exists B \subset \R$ such that $m(B) = 0$ and $f_n \rightarrow f$ pointwise on $A \setminus B$
    \item \underline{Uniformly on $A$} if $f_n, f$ are $\R$-valued and $\forall \epsilon > 0 \; \exists n_0$ such that $\forall x \in A$, $\card{f_n(x) - f(x)} \leq \epsilon$, for all $n \geq n_0$.
\end{enumerate}

Clearly $(c) \implies (b) \implies (a)$ but the reverse arrows are all false.
For example $f_n (x) = x^n \rightarrow 0$ pointwise a.e on [0,1] but not pointwise on [0,1], and $f_n (x) = \sin(\frac{x}{n}) \rightarrow 0$ converges pointwise on $\R$ but not uniformly.

\begin{prop}
    If $E \in \Le$ and $f,f_n: E \rightarrow [-\infty, \infty]$ with all $f_n$ being measurable and $f_n \rightarrow f$ pointwise on $E$, the $f$ is measurable.
\end{prop}

\begin{definition}[Simple Function]
    If $E$ measurable, then $\psi: E \rightarrow \R$ is called simple if it  is measurable, and takes only a finite number of values.
    Call these values $\{ c_j\}_{j=1}^{n}$, for some $n \geq 1$.
    Then if we call $E_j = \psi^{-1} (c_j) = \{ x \in E \mid \psi(x) = c_j \}$ then we have $E_j$ measurable $\forall j = 1 \hdots n$ and $E = \cup_{j=1}^{n} E_j$ disjoint.
    Also $\psi = c_j$ on $E_j$ so
    \[
        \boxed{\psi = \sum_{j=1}^{n} \chi_E c_j}
    \]
    In other words, simple functions are the same thing as finite linear combinations (with $\R$ coefficients) of characteristic functions of measurable sets.
\end{definition}

\underline{Approximation Lemma:} We have $E$ measurable and $f: E \rightarrow \R$ measurable.
Suppose $f$ is bounded, i.e $\exists C > 0$ such that $\card{f} \leq C$ then $\forall \epsilon \; \exists \phi_{\epsilon}, \psi_{\epsilon}$ simple functions on $E$ such that $\phi_{\epsilon} \leq f \leq \psi_{\epsilon}$  on $E$ and $0 \leq \psi_{\epsilon} - \phi_{\epsilon} \leq \epsilon$ on $E$.

\begin{prop}
    $E \subset \R$ measurable, $f: E \rightarrow [-\infty, \infty]$.
    Then $f$ is measurable $\iff \exists \{ \psi_n\}_{n=1}^{\infty}$, $\psi_{n}: E \rightarrow \R$ simple functions, $\psi_n \rightarrow f$ pointwise on $E$, and $\card{\psi_n} \leq \card{f}$ on $E$, for all $n$.
    If $f \geq 0$, we may choose $\psi_n$ such that $\psi_{n+1} \leq \psi_n$ on $E \; \forall n$.
\end{prop}

\begin{definition}[Null-Set]
    A set $A \subset \R$ with $\mstar{A} = 0$ is called a null-set.
\end{definition}

\begin{theorem}[Egorov's Theorem]
    For $E \in \Le$ with $m(E) < \infty$, Let $\{ f_n \}_{n=1}^{\infty}$ be measurable functions.
    $f_n : E \rightarrow [-\infty, \infty]$ which converge pointwise a.e to $f: E \rightarrow [-\infty, \infty]$ which is finite a.e on $E$ (i.e $f$ is $\R$-valued except for a null-set in $E$).
    Then $\forall \epsilon > 0$, $\exists F \subset E$ closed set, such that $m(E \setminus F) \leq \epsilon$ and $f_n \rightarrow f$ uniformly on $F$.
\end{theorem}

To start, observe that we may assume there are $E_0, E_0 \pr \subset E$ two null sets such that $f_n \rightarrow f$ pointwise on $E \setminus E_0$ and $f: E \setminus E_0 \pr \rightarrow \R$.
Thus, both of these hold on $E \setminus (\underset{\text{still a null set}}{E_0 \cup E_0 \pr} )$, and if we prove Egorov on $ E \setminus (E_0 \cup E_0 \pr)$ then this gives Egorov on $E$.
Thus, up to relabeling $E \rightsquigarrow E \setminus (E_0 \cup E_0 \pr)$, we shall assume form the start that
\[
    \boxed{f_n \rightarrow f \text{ pointwise on } E \text{ and } f: E \rightarrow \R}
\]
We already know that $f$ is measurable on $E$.

\begin{lemma}
    Suppose we are in this setting.
    Then, $\forall \eta > 0$, $\forall \delta > 0$, $\exists A \subset E, A \in \Le$, and $\exists N \geq 1$ such that $m(E \setminus A) \leq \delta$ and $\card{f_n - f} \leq \eta$ on $A$ for all $n \geq N$.
\end{lemma}

\begin{theorem}[Lusin's Theorem]
    Let $E \in \mathcal{L}$, $f: E \rightarrow [-\infty, \infty]$ be measurable and finite a.e, then $\forall \epsilon > 0$, $\exists F \subset E$ closed with $m(E \setminus F) \leq \epsilon$ and $\exists g: \R \rightarrow \R$ continuous, such that $f = g$ on $F$.
\end{theorem}


    \section{Integration}\label{sec:integration}
    \begin{definition}[Step Functions]
    Step functions are a special class of simple functions. $\phi: [a,b] \rightarrow \R$ is a step function if there exist finitely many disjoint intervals $\{ E_j\}_{j=1}^{n}$, $E_j \subset [a,b] \forall j$, $\cup_{j=1}^{n} E_j = [a,b]$, and $\exists c_j \in \R$, such that $\phi = \sum_{j=1}^{n} c_j \chi_{E_j}$.
\end{definition}

Observe that if $\phi$ is a step function then $\{E_j\}_{j=1}^{n}$ give us a partition $\mathcal{P}$ of $[a,b]$ and
\[
    \mathbf{L} (\phi, \mathcal{P}) = \sum_{j=1}^{n} c_j \ell (E_j) = \mathbf{U}(\phi, \mathcal{P})
\]
Where $\mathbf{L}$ and $\mathbf{U}$ are the lower Darboux sums defined in Riemann integration.
So for any partition $\mathcal{Q}$ of [a,b]
\[
    \sup_{Q} \mathbf{L} (\phi, \mathcal{Q}) \geq \mathbf{L} (\phi, \mathcal{P}) =  \mathbf{U}(\phi, \mathcal{P}) \geq \inf_{\mathcal{Q}}\mathbf{U}(\phi, \mathcal{Q}) \geq \sup_{Q} \mathbf{L} (\phi, \mathcal{Q})
\]
Hence they are equal, and $\phi$ is Riemann integrable and
\[
    \int_{a}^{b} \phi(x) dx =  \sum_{j=1}^{n} c_j \ell (E_j)
\]
One can prove that if $f$ is Riemann integrable on $[a,b]$, then
\begin{gather*}
    \sup \left\{ \int_{a}^{b} \phi (x) dx \mid \phi \text{ step function and } \phi \leq f \text{ on } [a,b] \right\} \\
    = \inf \left\{ \int_{a}^{b} \psi (x) dx \mid \psi \text{ step function and } \psi \geq f \text{ on } [a,b] \right\}
\end{gather*}

To define the \emph{Lebesgue Integral} we will proceed in steps.

\underline{Step 1:}

Suppose $\phi$ is a simple function, so $E \in \Le$, $\phi: E\rightarrow \R$ has the form $\phi = \sum_{j=1}^{n} a_j \chi_{E_j}$ where $a_j \in \R$ is distinct and $E_j \subset E$, $\cup_{j=1}^{n} E_j = E$ is a disjoint union.

Suppose $m(E) < \infty$, then we define the Legesbue integral as
\[
    \boxed{\int_{E} \phi = \int_E \phi (x) dx = \sum_{j=1}^{n} a_j m(E_j)}
\]

\begin{prop}
    $E \in \Le$ with $m(E) < \infty$, $\phi, \psi: E \rightarrow \R$ are simple functions then $\forall \alpha, \beta \in \R$,
    \[
        \int_{E} \alpha \phi + \beta \psi = \alpha \int_{E} \phi + \beta \intE \psi \tag{Linearity}
    \]
    Also, if $\phi \leq \psi$ on $E$, then
    \[
        \intE \phi \leq \intE \psi \tag{Monotonicity}
    \]
\end{prop}

\underline{Step 2:}

$E \in \Le$, $m(E) < \infty$, $f: E \rightarrow \R$ bounded.
We say that $f$ is Lebesgue integrable if $\mathbf{L} (f) = \mathbf{U}(f)$ where
\begin{align*}
    \mathbf{L}(f) &= \sup \left\{ \int_{a}^{b} \phi (x) dx \mid \phi \text{ step function and } \phi \leq f \text{ on } [a,b] \right\} \\
    \mathbf{U} (f) &= \inf \left\{ \int_{a}^{b} \psi (x) dx \mid \psi \text{ step function and } \psi \geq f \text{ on } [a,b] \right\}
\end{align*}

\begin{theorem}
    $a,b \in \R$, $a < b$, $f: [a,b] \rightarrow \R$ a bounded function.
    Suppose $f$ is Riemann integrable, then $f$ is Lebesgue integrable on $[a,b]$ and the two integrals are equal.
\end{theorem}

\begin{theorem}
    $E \in \Le$ with $m(E) < \infty$, $f: E \rightarrow \R$ measurable and bounded, then $f$ is Lebesgue integrable over $E$.
\end{theorem}

\begin{theorem}
    $E \in \Le, m(E) < \infty$, $f,g: E \rightarrow \R$ bounded measurable functions.
    $\forall \alpha, \beta \in \R$,
    \[
        \intE (\alpha f + \beta g) = \alpha \intE f + \beta \intE g
    \]
    Also, if $f \leq g$ on $E$ then $\int_{E} f \leq \int_{E} g$
\end{theorem}

\begin{corollary}[Chopping]
    $E \in \Le$, $m(E) < \infty$, $f: E \rightarrow \R$ bounded and measurable.
    If $A, B \subset E$, $A, B \in \Le$, $A \cap B = \emptyset$, then
    \[
        \boxed{ \int_{A \cup B} f = \int_{A}f + \int_{B} f}
    \]
\end{corollary}

\begin{prop}[Extremely Useful Inequality]
    $E \in \Le$, $m(E) < \infty$, $f: E \rightarrow \R$ bounded and measurable, then
    \[
        \boxed{ \left\lvert \intE f \right\rvert \leq \intE \card{f}}
    \]
\end{prop}

\begin{prop}
    $E \in \Le$, $m(E) < \infty$, $f_n: E \rightarrow \R$ bounded measurable.
    If $f_n \rightarrow f$ \underline{uniformly} on $E$, then
    \[
        \lim_{n \rightarrow \infty} \intE f_n = \intE f
    \]
\end{prop}

\begin{theorem}[Bounded Convergence Theorem]
    $E \in \Le$, $m(E) < \infty$, $f_n: E \rightarrow \R$ bounded, $f_n \rightarrow f$ pointwise on $E$.
    Suppose that $\exists M > 0$ such that $\card{f_n} \leq M$ on $E$, $\forall n$.
    Then
    \[
        \lim_{n \rightarrow \infty} \intE f_n = \intE f
    \]
\end{theorem}

\underline{Step 3:}

\begin{definition}[Finite Support]
    $E \in \Le$, not necessarily with $m(E) < \infty$.
    $f: E \rightarrow [-\infty, \infty]$ measurable.
    We say that $f$ has \emph{finite support} if its support $\text{Supp}(f) = \{ x \in E: f(x) \neq 0 \} \in \Le$ satisfies $m(\text{Supp}(f)) < \infinity$.
    In other words, $f$ is zero outside a measurable subset with finite measure.
    In this case, if $f: E \rightarrow \R$ bounded and measurable, $m(E)$ may be infinite, and if $f$ has finite support, we define
    \[
        \intE f \coloneqq \int_{\text{Supp}(f)} f
    \]
\end{definition}

Now, for $E \in \Le$ and $f: E \rightarrow [0, \infty]$ measurable non-negative function, define
\[
    \intE f = \sup \left\{ \intE h \mid h: E \rightarrow \R \text{ bounded, measurable of finite support with } 0 \leq h \leq f \text{ on } E \right\}
\]

\begin{theorem}[Chebyshev's Inequality]
    $E \in \Le$, $f: E \rightarrow [0,\infty]$ measurable.
    Then $\forall \lambda > 0$.
    \[
        \boxed{m \{ f \geq \lambda \} \leq \frac{1}{\lambda} \intE f}
    \]

\end{theorem}

\begin{corollary}
    $E \in \Le$, $f: E \rightarrow [0, \infty]$ measurable, then
    \[
        \intE f = 0 \iff f = 0 \text{ a.e} \text{ on } E
    \]
\end{corollary}

Linearity and Monotonicity also apply to step 3 of the definition.

\begin{prop}[Fatou's Lemma]
    $E \in \Le$, $f_n: E \rightarrow [0, \infty]$ measurable, suppose $f_n \rightarrow f$ pointwise a.e on $E$.
    Then
    \[
        \intE f \leq \liminf_{n \rightarrow \infty} \intE f_n
    \]
\end{prop}

\begin{theorem}[Monotone Convergence Theorem]
    $E \in \Le$, $f_n: E \rightarrow [0. \infty]$ measurable with $\{ f_n \}$ increasing (i.e $f_n \leq f_{n+1} \text{ on } E \; \forall n \geq 1$ ).
    Assume $f_n \rightarrow f$ pointwise a.e on $E$.
    Then
    \[
        \boxed{ \intE f = \lim_{n \rightarrow \infty} \intE f_n}
    \]
\end{theorem}

\begin{definition}
    $E \in \Le$, $f: E \rightarrow [0, \infty]$ measurable.
    We say that $f$ is \underline{integrable} over $E$ if $\intE f < \infty$.
\end{definition}

\begin{prop}
    $f$ integrable $\implies f $ finite a.e on $E$.
\end{prop}

\begin{prop}[Beppolevi's Lemma]
    $E \in \Le$, $f_n: E \rightarrow [0,\infty]$ measurable with $f_n \leq f_{n+1}$ $\forall n$.
    Suppose $\exists C > 0$ such that $\intE f_n \leq C $ $\forall n$.
    Then $f_n \rightarrow f$ pointwise on $E$, $f: E \rightarrow [0, \infty]$ measurable and finite a.e on $E$, and $\lim_{n \rightarrow \infty} \intE f_n = \intE f < \infty$.
\end{prop}

\underline{Step 4}:

Now for general functions.
$E \in \Le, f: E \rightarrow [-\infty, \infty]$, measurable.
Then $f^{+}, f^{-}: E \rightarrow [0,\infty]$ are measureable and
\[
    \begin{cases}
        f &= f^{+} - f^{-} \\
        \card{f} &= f^{+} + f^{-}
    \end{cases}
\]
on $E$.

\begin{lemma}
    $\card{f}$ integrable on $E \iff f^{+} \text{ and } f^{-}$ integrable on $E$
\end{lemma}

\begin{definition}
    $E \in \Le, f: E \rightarrow [-\infty, \infty]$ measureable.
    We say that $f$ is integrable if $\card{f}$ integrable.
    Then let
    \[
        \intE f = \intE f^{+} - \intE f^{-} \in \R
    \]
\end{definition}
This clearly agrees with the earlier definition if $f \leq 0$, since then $f^{-} = 0$.

\begin{prop}
    $f$ integrable on $E \implies f $ finite a.e on $E$, and $\forall E_0 $ null set in $E$,
    \[
        \intE f = \int_{E \setminus E_0} f
    \]
\end{prop}

\begin{prop}
    $E \in \Le$, $f: E \rightarrow [-\infty, \infty]$ measurable.
    Suppose $g: E \rightarrow [0,\infty]$ measurable such that $g$ integrable on $E$ and $\card{f} \leq g$ on $E$.
    Then $f$ also integrable, and
    \[
        \lvert \intE f \rvert \leq \intE \card{f}
    \]
\end{prop}

Now if $f,g$ are integrable over $E$, $f+g$ can only be defined at points where $f$ and $g$ are finite.
But we know that $E_0 = \{ f = \pm \infty \} \cup \{ g = \pm \infty \}$ is a null set, so on $E \setminus E_0$ we define $f+g$, we will show that $f+g$ is integrable on $E\setminus E_0$, and then define $\intE (f+g) \coloneqq \int_{E \setminus E_0} (f+g)$.

Linearity, monotonicity, and chopping also hold true for this definition of the Lebesgue integral.

\begin{theorem}[Dominated Convergence]
    $E \in \Le$, $f_n: E \rightarrow [-\infty, \infty]$ measurable.
    Suppose $f_n \rightarrow f$ pointwise a.e on $E$ and $\card{f_n} \leq g$ on $E$ $\forall n$ for some $g$ integrable on $E$.
    Then $f$ integrable and
    \[
        \intE f = \lim_{n \rightarrow \infty} \intE f_n
    \]
\end{theorem}

    \section{Lebesgue Measure in $\R^n$}\label{sec:lebesgue-measure-in-R^n}
    We now briefly extend the theory of Lebesgue measure to $\R^n$, $n \geq 1$.
To start, using open sets in $\R^n$, one defines the Borel $\sigma$-algebra $\mathcal{B}$ on $\R^n$.

To define the Lebesuge outer measure, the role of intervals is played by rectangles (or boxes).
A box $I$ in $\R^{n}$ is a product of intervals $I = I_1 \times \hdots \times I_n$ where each $I_j \subset \R$ is an interval.
Then $I$ open $\iff I_j$ open $\forall j$, $I$ bounded $\iff I_j$ bounded $\forall j$.

The analogue of the length $\ell (I)$ is now the Volume Vol$(I) = \prod_{j=1}^{n} \ell (I_j) \in [0,\infty]$.
Clearly Vol$(I) < \infty \iff I$ bounded.

If $A \subset \R^n$, let $\mathcal{C}_A = \left\{ \{ I_j\}_{j=1}^{\infty} \mid I_j \text{ bounded open boxes with } A \subset \cup_{j=1}^{\infty} I_j  \right\}$

Again, we let
\[
    m^{\star} (A) \coloneqq \inf_{\{ I_j \} \in \mathcal{C}_A } \sum_{j=1}^{\infty} \text{Vol}(I_j)
\]
All of the classic properties of the case when $\R = 1$ also hold for $\R^n$.

\underline{Product Sets:}

\begin{lemma}
    $A \subset \R^a$, $B \subset \R^b$ any sets, $A \times B \subset \R^{a+b}$, then $m^{\star}(A \times B) \leq m^{\star} (A) m^{\star}(B)$, with the convention that $0 \times \infty = 0$
\end{lemma}

\begin{prop}
    If $A \subset \R^{a}$, $A \in \Le$, $B \subset \R^b$, $B \in \Le$, then $A \times B \subset \R^{a+b}$ is measurable.
\end{prop}

\begin{definition}[Slices]
    $E \subset \R^n$, $(n \geq 2)$, suppose $E \in \Le$.
    A slice of $E$ is a set of this form: write $\R^n = \R^a \times \R^b$, $a+b = n$.
\end{definition}

Pick any $x \in \R^{a}$ and let $E_x = \text{ slice } = \{ y \in \R^{b} \mid (x,y) \in E \}$.
There is a problem: $E \in \Le \centernot\implies E_x \in \Le$

\begin{theorem}[Fubini's Theorem]
    Suppose $f: \R^n = \R^a \times \R^b \rightarrow [-\infty, \infty]$ is integrable with respect to Lebesgue on $\R^n$.
    Then for a.e $y \in \R^b$, the slice $f(\cdot, y)$ is integrable in $\R^a$ and the function $y \mapsto \int_{\R^a} f(x,y)dx$ is integrable in $\R^b$, we also have
    \[
        \int_{\R^n} f = \int_{\R^b} \left( \int_{\R^a} f(x,y)dx \right)dy
    \]
\end{theorem}

The theorem is symmetric in $x$ and $y$ so we also have $\int_{\R^n} f = \int_{\R^a} \left( \int_{\R^b} f(x,y)dy \right)dx$ and for a.e $x \in \R^a$, $f(x,\cdot)$ is integrable in $\R^b$ and $x \mapsto \int_{\R^b} f(x,y)dy$ is integrable in $\R$.

\begin{corollary}[Tonelli's Theorem]
    $f: \R^n = \R^a \times \R^b \rightarrow [0, \infty]$ is measurable nonegative function.
    Then for a.e $y \in \R^b$, $f(\cdot, y)$ is measurable on $\R$ and $y \mapsto \int_{\R^a} f(x,y) dx$ is measurable on $\R^b$, and
    \[
        \int_{\R^n} f = \int_{\R^b} \left(  \int_{\R^a} f(x,y)dx \right)dy
    \]
\end{corollary}

Usually, one applies Tonelli to $\card{f}$, where $f$ is measurable on $\R^n$, so that $\int_{\R^n} \card{f} = \int_{\R^b} \left( \int_{\R^a} \card{f} (x,y) dx \right)dy$, so if the LHS is finite, so is the RHS, hence $f$ is integrable in $\R^n$, so Fubini applies to $f$,
\[
    \int_{\R^n} f = \int_{\R^b} \left( \int_{\R^a} f(x,y)dx \right)dy
\]

\begin{corollary}[Cavalieri's Formula]
    $E \subset \R^n = \R^a \times \R^b$ measurable, then for a.e $y \in \R^b$, $E_y$ is measurable in $\R^a$.
    Also $y \mapsto m(E_y)$ is a measurable function and
    \[
        \boxed{m(E) = \int_{\R^b} m(E_y) dy}
    \]
\end{corollary}

\begin{corollary}
    If $A \subset \R^a$, $A \in \Le$, $B \subset \R^b$, $B \in \Le$, then $A \times B \subset \R^{a+b}$ is measurable (we already knew that) and $m(A \times B) = m(A)m(B)$.
\end{corollary}

\end{document}