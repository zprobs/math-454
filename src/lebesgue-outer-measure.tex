We are hoping to measure the size of subsets of $\mathbb{R}$.
Ideally we would like to find or construct a function
\[
    m: \pwr{R} \rightarrow \mathbb{R}_{\geq 0} \cup \{ + \infty \} = [0, \infty]
\]
Which satisfies the following measure requirements:
\begin{enumerate}
    \item If $I=[a,b]$ or $(a,b)$ or $[a,b)$, or $(a,b]$, $a,b \in \mathbb{R}, a \leq b$ then $m(I) = b-a = $ measure of interval
    \item $m$ is translation invariant.
    i.e if $E \subset \mathbb{R}$ and $x \in \mathbb{R}$, let $E + x = \{ y+x \mid y \in E \}$ then $m(E+x)$ = m(E)
    \item If $\{ E_j \}_{j=1}^{n}$ is a finite collection of pairwise disjoint $E_j \subset \mathbb{R}$ then
    \[
        m \left( \cup_{j=1}^{n} E_j \right) = \sum_{j=1}^{n} m(E_j)
    \]
    \item The same as (3) except for $n = \infty$
\end{enumerate}

\begin{theorem}
    There is no such $m$ satisfying all 4 requirements
\end{theorem}

The proof for this will come later.
The solution for this is that we do not try to measure all subsets of $\mathbb{R}$.
So we have $m: \pwr{R} \rightarrow [0, \infty]$ but now we will just be happy with $m: \mathcal{A} \rightarrow [0,\infty]$ where $\mathcal{A}$ is a $\siga$ which has enough elements.
For example $\mathcal{A} > \mathcal{B}$.

We will follow H. Lebesgue as we proceed in two steps.

\underline{Step 1:} construct Lebesgue outer measure $m^{\star}: \pwr{R} \rightarrow [0, \infty]$ satisfying requirements 1,2, and 3.

\underline{Step 2:} Use $m^{\star}$ to define $\mathcal{A}$ and let $m \subset m^{\star} \mid \mathcal{A}$

To create this Lebesgue outer measure on $\mathbb{R}$ we satisfy a weakened version of requirement (3) that can be called (3w).
For any countably infinite collection $\{ E_j \}_{j=1}^{\infty}$ of arbitrary subsets $E_j \subset \mathbb{R}$
\[
    m^{\star}(\cup_{j=1}^{\infty} E_j) \leq \sum_{j=1}^{\infty} m(E_j)
\]
\begin{theorem}[Lebesgue Outer Measure]
    There is a map $m^{\star}: \pwr{R} \rightarrow \rinf $ that satisfies the measure requirements 1, 2, and 3w.
\end{theorem}

This $m^{\star}$ is called the Lebesgue outer measure on $\mathbb{R}$.

How do we define outer measure $m^{\star}(A)$?

Observe that any $A \subseteq \mathbb{R}$ can be covered by some countable infinite collection $\{ I_j \}_{j=1}^{\infty}$ of bounded open intervals, which are allowed to be empty, but we do not assume that $I_j$ be pairwise disjoint.

For example: $I_j = (-j, j), \; j=1,2,3\hdots$

Let
\[
    \mathcal{C}_A = \{ \{I_j\}_{j=1}^{\infty} \mid I_j \text{ bounded open intervals such that } A \subset \cup_{j=1}^{\infty} I_j \}
\]
$\mathcal{C}_A \neq \emptyset$ by our example so for each $\{ I_j \} \in \mathcal{C}_A$, I can consider
\[
    \sum_{j=1}^{\infty} \ell(I_j) \in \rinf \tag{$\ell$ denotes length}
\]

\begin{definition}[Outer Measure]
    \[
        \boxed{m^{\star}(A) \coloneqq \inf_{\{ I_j \} \in \mathcal{C_A}} \sum_{j=1}^{\infty} \ell (I_j)  } \in \rinf
    \]
\end{definition}

This defines a map $m^{\star}: \pwr{R} \rightarrow \rinf$

\underline{Simple Properties:}

\begin{itemize}
    \item \emph{Monotonicity}:
    If $A \subseteq B$ then $m^{\star}(A) \leq m^{\star}(B)$.
    Indeed by definition $\mathcal{C}_B \subseteq \mathcal{C}_A$ hence the infimum over $\mathcal{C}_B$ is $\geq$ than the infimum over $\mathcal{C}_A$.
    \item \emph{Empty Set}: $\mstar{\emptyset} = 0$.
    Given any $1 > \epsilon > 0$, let $I_j = (-\epsilon^{j}, \epsilon^j), \; j=1,2,\hdots$
    $\{I_j\} \in \mathcal{C}_{\emptyset}$ and $\sum_{j=1}^{\infty} \ell (I_j) = 2 \sum_{j=1}^{\infty} \epsilon^j = \frac{2\epsilon}{1-\epsilon} $ from the geometric series going to zero so $\mstar{\emptyset} \leq \frac{2\epsilon}{1-\epsilon} \; \forall 0 < \epsilon < 1$
    \item If $A \in \mathbb{R}$ is finite or countable infinite then $\mstar{A} = 0$.
    Indeed enumerate all elements of $A$ by $\{ a_j \}_{j=1}^{\infty}$.
    (If $A$ is finite say $\card{A} = n$ let $a_j = a_n$ for all $j > n$).
    For any $0 < \epsilon < 1$, let $I_j = \left( -\epsilon^j + a_j, a_j + \epsilon^j \right)$ so $A \subseteq \cup_{j=1}^{\infty} I_j$ and $\isum{j} \ell (I_j) = \frac{2\epsilon}{1-\epsilon}$ hence as before, $\mstar{A} = 0$.
    For example $\mstar{\mathbb{Q}} = 0$
\end{itemize}

We will now prove that the Lebesgue outer measure satisfies 1, 2, and 3w of the measure requirements.

\underline{Proof of Property 1}:
i.e $\mstar{I} = \ell(I)$ for any interval $I \subseteq \mathbb{R}$

Assume that $I = [a,b]$, $a < b$ are finite numbers.
Assume that $I$ is a bounded closed interval.
Our goal is to show that $\mstar{I} = b -a$.
One direction of inequality is easy to prove, the other is quite tedious and will be left out.

For any $\epsilon > 0$ let $I_1 = (a-\epsilon, b + \epsilon) > I$, let $I_j = \emptyset, j \geq 2$ so $\{I_j \} \in \mathcal{C}_I \Rightarrow \mstar{I} \leq \isum{j} \ell (I_j) = b - a + 2 \epsilon$.
Let $\epsilon \rightarrow 0$ and we obtain $\mstar{I} \leq b-a$.

\underline{Proof of Property 2}:
i.e $\forall A \subset \mathbb{R}, \forall x \in \mathbb{R}$, $\mstar{A+x} = \mstar{A}$

$\mathcal{C}_A$ and $\mathcal{C}_{A+x}$ are naturally in bijection via $\{ I_j \} \leftrightarrow \{ I_j + x \}$.
Furthermore $\ell (I_j + x) = \ell (I_j)$
\begin{align*}
    \mstar{A+x} &= \inf_{\{I_j +x\} \in \mathcal{C}_{A+x}} \isum{j} \ell (I_j + x) \\
    &= \inf_{\{I_j\} \in \mathcal{C}_{A}} \isum{j} \ell (I_j) = \mstar{A}
\end{align*}

\underline{Proof of Property 3w}:
i.e If $\{ E_j \}_{j=1}^{n}$ is a finite collection of pairwise disjoint $E_j \subset \mathbb{R}$ then $\mstar{\cup_{j=1}^{n} E_j} = \sum_{j=1}^{n} \mstar{E_j}$

If $\mstar{E_j} = +\infty$ for some $j$, then the property holds.
We may assume that $\mstar{E_j} < + \infty \; \forall j$.
Let $\epsilon > 0$.
By the definition of infimum, for each $j \geq 0$, there is
\[
    \{ I_{j,k} \}_{k=1}^{\infty} \in \mathcal{C}_{E_j} \text{ such that } \isum{k} \ell (I_{j,k}) < \mstar{E_j} + \epsilon 2^{-j}
\]
Thus $\{ I_{j,k} \}_{k=1}^{\infty}$ is still countable and it covers $\cup_{j=1}^{\infty} E_j$ meaning it belongs to $\mathcal{C}_{\cup_{j=1}^{\infty}} E_j$, so by definition
\[
   \mstar{\bigcup_{j=1}^{\infty} E_j} \leq \sum_{j=1}^{\infty} \isum{k} \ell (I_{j,k}) < \sum_{j=1}^{\infty} (\mstar{E_j} + \epsilon 2^{-j}) = \sum_{j=1}^{\infty} \mstar{E_j} + \epsilon
\]
Then let $\epsilon \rightarrow 0$.
Clearly, by taking all $E_j = \emptyset$ except finitely many, we have the same subadditivity 3w for finite collections.

\begin{corollary}
    $\mstar{[0,1] \cap (\mathbb{R} \setminus \mathbb{Q})} = 1 = \ell([0,1])$
\end{corollary}
\begin{proof}
    \begin{align*}
        \mstar{[0,1] \cap (\mathbb{R} \setminus \mathbb{Q})} &\leq \mstar{[0,1]} = 1 \\
        &\leq \mstar{[0,1] \cap (\mathbb{Q})} + \mstar{[0,1] \cap (\mathbb{R} \setminus \mathbb{Q})} \\
        &\leq 0 + 1
    \end{align*}
\end{proof}

\begin{corollary}
    $\mathbb{R} \setminus \mathbb{Q}$ is uncountable
\end{corollary}
\begin{proof}
    If not, then
    \[
        \mstar{\mathbb{R} \setminus \mathbb{Q}} = 0 \geq \mstar{[0,1] \cap (\mathbb{R} \setminus \mathbb{Q})} = 1 \qedhere
    \]
\end{proof}