%! Author = zachprobst
%! Date = 2021-09-21

% Preamble
\documentclass[11pt]{article}

% Packages
\usepackage{amsmath}
\usepackage{amsfonts}
\usepackage{amsthm}
\usepackage{mathtools}
\usepackage[makeroom]{cancel}
\usepackage{tcolorbox}

% Setup
\title{Homework 2}
\author{Zachary Probst}
\setlength{\parindent}{0pt}
\setlength{\parskip}{\baselineskip}
\DeclarePairedDelimiter{\ceil}{\lceil}{\rceil}
\DeclarePairedDelimiter{\card}{\lvert}{\rvert}
\newtcolorbox{mybox}{colback=blue!5!white,colframe=blue!75!black}


\newtheorem{theorem}{Theorem}[section]
\newtheorem*{prop}{Proposition}
\newtheorem{corollary}{Corollary}[theorem]
\newtheorem*{*corollary}{Corollary}
\newtheorem{lemma}[theorem]{Lemma}
\newtheorem{definition}{Definition}[section]
\newtheorem*{remark}{Remark}

% Commands
\newcommand{\siga}{\sigma\text{-algebra}}
\newcommand{\pwr}[1]{\mathcal{P}(\mathbb{#1})}
\newcommand{\rinf}{\mathbb{R}_{\geq 0} \cup \{ + \infty \}}
\newcommand{\mstar}[1]{m^{\star}\left(#1\right)}
\newcommand{\isum}[1]{\sum_{#1=1}^{\infty}}
\newcommand{\pr}{^{\prime}}

% Document
\begin{document}

    \maketitle

    \begin{mybox}
        \textbf{Problem 1:}
        Show that a set $E \subset \mathbb{R}$ is measurable if and only if for every interval $I \subset \mathbb{R}$ we have
        \begin{equation}
            \mstar{I} = \mstar{I \cap E} + \mstar{I \cap E^c}\label{eq:equation}
        \end{equation}
        To prove the nontrivial direction, we assume~(\ref{eq:equation}) and need to show that given any $A \subset \mathbb{R}$ with $\mstar{A} < \infty$ we have
        \begin{equation}
            \mstar{A} \geq \mstar{A \cap E} + \mstar{A \cap E^c}\label{eq:equation2}
        \end{equation}
    \end{mybox}

    First we will show that we have
    \[
        \mstar{U} = \mstar{U \cap E} + \mstar{U + E^c}
    \]
    for any open set $U \subset \mathbb{R}$.
    Since $U$ is open, it can be written as a countable union of disjoint open intervals $U = \cup_{k=1}^{\infty} I_k$.
    Let $n \in \mathbb{N}$ be any natural number.
    Define $F_n \coloneqq \cup_{k=1}^{n} I_k$.
    Since $F_n$ is a countable union of intervals and intervals are measurable, $F_n$ is measurable by \emph{Proposition 7}.

    By \emph{Proposition 6} we have:
    \[
        \mstar{E \cap F_n} = \sum_{k=1}^{n} \mstar{E \cap I_k}
    \]
    \begin{align*}
        \mstar{F_n} &= \mstar{\bigcup_{k=1}^{n} I_k} \\
        &= \sum_{k=1}^{n} \mstar{I_k} \tag{Proposition 6} \\
        &= \sum_{k=1}^{n} \mstar{I_k \cap E} + \mstar{I_k \cap E^c}\tag{Equation~\eqref{eq:equation}} \\
        &= \mstar{F_n \cap E} + \mstar{F_n \cap E^c} \tag{Proposition 6}\\
    \end{align*}
    Now take the limit as $n \rightarrow \infty$ on both sides of this equation and we arrive at
    \begin{equation}
        \mstar{U} = \mstar{U \cap E} + \mstar{U \cap E^c}\label{eq:equation3}
    \end{equation}

    Next let $\epsilon > 0$.
    By the definition of infimum, There exists some $U = \cup_{k=1}^{\infty} I_k \in \mathcal{C}_A$ such that $\sum_{k=1}^{\infty} \ell (I_k) < \mstar{A} + \epsilon$.
    $U$ is a countable union of nonempty open bounded intervals, so it must be its own minimal cover.
    This means $\mstar{U} = \sum_{k=1}^{\infty} \ell (I_k)$.
    Using monotonicity and the fact that $A \subset U$
    \[
        \mstar{A} \leq \mstar{U} < \mstar{A} + \epsilon
    \]
    Since we have $\mstar{A} < \infty$ we can subtract $\mstar{A}$ to arrive at
    \[
        0 \leq \mstar{U} - \mstar{A} < \epsilon
    \]
    For all $\epsilon$ so $\mstar{U} = \mstar{A}$.

    Now to conclude:
    \begin{align*}
        \mstar{A} &= \mstar{U} \\
        &= \mstar{U  \cap E} + \mstar{U \cap E^c} \tag{Equation~\eqref{eq:equation3}} \\
        &\geq \mstar{A \cap E} + \mstar{A \cap E^c} \tag{$A \subset U$}
    \end{align*}
    So putting this result together with the trivial inequality in the other direction
    \[
        \mstar{A} = \mstar{A \cap E} + \mstar{A \cap E^c}
    \]

    \clearpage
    \begin{mybox}
        \textbf{Problem 2:}
        Let $E \subset \mathbb{R}$ be any set.
        Show that there is a $G_\delta$ set $F \supset E$ (in particular $F$ is measurable) with
        \[
            \mstar{F} = \mstar{E}
        \]
    \end{mybox}
    By monotonicity we have $\mstar{E} \leq \mstar{F}$ for any $F$.

    To show the reverse inequality, $\forall n \in \mathbb{N}$, from the definition of infimum, there must exist a collection $\{ I_{n_k} \}_{k=1}^{\infty}$ of open intervals such that $\{ I_{n_k} \}_{k=1}^{\infty} \in \mathcal{C}_E$ and
    \[
        \mstar{E} \leq \sum_{k=1}^{\infty} \ell (I_{n_k}) < \mstar{E} + \frac{1}{n}
    \]
    Define $F$ to be
    \[
        F \coloneqq \bigcap_{n=1}^{\infty} \bigcup_{k=1}^{\infty} I_{n_k}
    \]
    $F$ is a countable intersection since the natural numbers are countable and it is an open set since it is composed of intervals.
    Therefore $F$ is a $G_{\delta}$.

    For any $\epsilon > 0$, let $N \in \mathbb{N}$ be such that $\frac{1}{N} < \epsilon$.
    \begin{align*}
        \mstar{F} &= \bigcap_{n=1}^{\infty} \bigcup_{k=1}^{\infty} I_{n_k} \\
        &\leq \mstar{\bigcup_{k=1}^{\infty} I_{N_k}} \tag{Monotonicity} \\
        &\leq \sum_{k=1}^{\infty} \mstar{I_{N_k}} \tag{Subadditivity} \\
        &< \mstar{E} + \frac{1}{N} \\
        &< \mstar{E} + \epsilon \\
    \end{align*}
    This must mean that $\mstar{F} = \mstar{E}$

    \clearpage
    \begin{mybox}
        \textbf{Chapter 2 Problem 10}
        Let $A$ and $B$ be bounded sets for which there is an $\alpha > 0$ such that $\card{a-b} \geq \alpha$ for all $a \in A, b \in B$.
        Prove that $\mstar{A \cup B} = \mstar{A} + \mstar{B}$
    \end{mybox}

    $A \cap B = \emptyset$ because otherwise there would be some $a \in A, b \in B$ such that $a = b \Rightarrow \card{a-b} = 0$ but $\card{a-b} > 0$ by definition.

    Since $A$ and $B$ are disjoint, any minimal collection of bounded open intervals $I_k$ that cover $A \cup B$ can be split up into two disjoint collections $I_k^{A} \supseteq A$ and $I_k ^{B} \supseteq B$ where $I_k^{A} \cap I_k ^{B} = \emptyset \; \forall k$

    \begin{align*}
        \mstar{A \cup B} &= \inf_{\{I_k\} \in \mathcal{C}_{A \cup B}} \sum_{k=1}^{\infty} \ell (I_k) \\
        &= \inf_{\{I_k^{A} \cup I_k^{B}\} \in \mathcal{C}_{A \cup B}} \sum_{k=1}^{\infty} \ell (I_k^{A}) + \sum_{k=1}^{\infty} \ell (I_k^{B})  \\
        &= \inf_{\{I_k^{A}\} \in \mathcal{C}_A} \sum_{k=1}^{\infty} \ell (I_k^{A}) + \inf_{\{I_k^{B}\} \in \mathcal{C}_B} \sum_{j=1}^{\infty} \ell (I_k^{B})  \\
        &= \mstar{A} + \mstar{B}
    \end{align*}

    \clearpage
    \begin{mybox}
        \textbf{Chapter 2 Problem 14:}
        Show that if a set $E$ has positive outer measure, then there is a bounded subset of $E$ that also has positive outer measure.
    \end{mybox}

    Define an open interval $I_j \coloneqq (-j, j) \cap E$.
    We claim that there is some $N \in \mathbb{N}$ such that $\mstar{I_N} > 0$.
    If there were no such $N$, then $\mstar{I_j} = 0 \; \forall j$ and we could let $j \rightarrow \infty$
    \begin{align*}
        \mstar{E} &= \mstar{\bigcup_{j=1}^{\infty} I_j}\\
        &\leq \isum{j} \mstar{I_j} \tag{Countable Subadditivity} \\
        &= 0
    \end{align*}
    But this is a contradiction since $\mstar{E} > 0$.
    Therefore there exists an $I_N$ bounded subset of $E$ that also has positive outer measure.


\end{document}