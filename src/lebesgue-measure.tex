We can now take $m^{\star}$ and restrict it to $\mathcal{L}$. $m^{\star} \mid_{\mathcal{L}}$.
\begin{definition}[Lebesgue Measure]
    This Lebesgue Measure is a function
    \[
        m \coloneqq m^{\star} \mid_{\mathcal{L}} : \mathcal{L} \rightarrow \rinf
    \]
\end{definition}
This means that for $E \in \mathcal{L}$ we define $m(E) = \mstar{E}$.
Clearly, $m$ satisfies the measurability requirements 1, 2, \& 3 as we have proved earlier.
It also satisfies requirement 4 which was requirement 3 for countably infinite sets.

\begin{prop}
    If $\{ E_j \}_{j=1}^{\infty}$ is a countably infinite collection of pairwise disjoint sets $E_j \in \mathcal{L}$ (possibly empty), then $\cup_{j=1}^{\infty} E_j \in \mathcal{L}$ and
    \[
        m\left(\bigcup_{j=1}^{\infty} E_j\right) = \sum_{j=1}^{\infty} m(E_j)
    \]
\end{prop}

\begin{proof}
    We proved earlier that $\cup_{j=1}^{\infty} E_j \in \mathcal{L}$ and that
    \[
        m\left(\bigcup_{j=1}^{\infty} E_j\right) \leq \sum_{j=1}^{\infty} m(E_j)
    \]
    For the opposite inequality, for each $n$ we proved earlier that
    \[
        m\left(\bigcup_{j=1}^{n} E_j\right) = \sum_{j=1}^{n} m(E_j)
    \]
    But $\cup_{j=1}^{n} E_j \subset \cup_{j=1}^{\infty} E_j$, hence
    \[
        m\left(\bigcup_{j=1}^{\infty} E_j\right) \geq m\left(\bigcup_{j=1}^{n} E_j\right) =  \sum_{j=1}^{n} m(E_j) \;\; \forall n
    \]
    Take the limit as $n \rightarrow \infty$ to get
    \[
        m\left(\bigcup_{j=1}^{\infty} E_j\right) \geq \sum_{j=1}^{\infty} m(E_j)
    \]
    As desired.
    This argument shows that measurability requirement 3 and 3w together imply 4.
\end{proof}