%! Author = zachprobst
%! Date = 2021-09-11

% Preamble
\documentclass[11pt]{article}

% Packages
\usepackage{amsmath}
\usepackage{amsfonts}
\usepackage{amsthm}
\usepackage{mathtools}
\usepackage{tcolorbox}
\DeclarePairedDelimiter{\ceil}{\lceil}{\rceil}

% Setup
\title{Homework 1}
\author{Zachary Probst}
\setlength{\parindent}{0pt}
\newtcolorbox{mybox}{colback=blue!5!white,colframe=blue!75!black}

\newtheorem{theorem}{Theorem}[section]
\newtheorem{corollary}{Corollary}[theorem]
\newtheorem{lemma}[theorem]{Lemma}
\newtheorem{definition}{Definition}[section]
\newtheorem*{remark}{Remark}

% Commands
\newcommand{\siga}{\sigma\text{-algebra}}
\newcommand{\pwr}[1]{\mathcal{P}(\mathbb{#1})}
\newcommand{\pr}{^{\prime}}

% Document
\begin{document}

    \maketitle

    \begin{mybox}
        \textbf{Chapter 1 Problem 33:} \\
        Show that the Nested Set Theorem is false if $F_1$ is unbounded.
    \end{mybox}

    Let $\{ F_n \}_{n=1}^{\infty} = \{ [n,\infty), n \in \mathbb{N} \}$.
    We have $F_1 = [1, \infty)$ so $F_1$ is unbounded.\\

    Assume that the Nested Set Theorem holds for descending countable collections of nonempty closed sets.
    Then there is some $x \in \mathbb{R}$ such that
    \[
        x \in \cap_{n=1}^{\infty} F_n
    \]
    Let $n\pr = \ceil{x}$ be the smallest integer larger than $x$.
    Then clearly $x \notin [n\pr, \infty)$ and so $x \notin F_{n\pr}$.
    But this is a contradiction so it must be that
    \[
        \cap_{n=1}^{\infty} F_n = \emptyset
    \]
    And the Nested Set Theorem does not hold for $F_1$ unbounded.\\

    \clearpage

    \begin{mybox}
        \textbf{Chapter 2 Problem 1:} \\
        Prove that if there is a set $A$ in the collection $\mathcal{A}$ with $A \subseteq B$, then $m(A) \leq m(B)$.
        This property is called \emph{monotonicity}.
    \end{mybox}

    If we have $A \subseteq B$ then we can write $B = A \cup (B \setminus A)$
    \begin{align*}
        \Rightarrow m(B) &= m(A \cup (B \setminus A)) \\
        &= m(A) + m(B \setminus A) \tag{by countable additivity} \\
        &\geq m(A)
    \end{align*}

    \clearpage

    \begin{mybox}
        \textbf{Chapter 2 Problem 2:} \\
        Prove that if there is a set $A$ in the collection $\mathcal{A}$ for which $m(A) < \infty$, then $m(\emptyset) = 0$.
    \end{mybox}

    Assume that $m(\emptyset) > 0$.
    Define a countable collection of sets $\{ E_k \}_{k=1}^{\infty}$ where
    \[
        E_k = \begin{cases}
                  A & k = 1 \\
                  \emptyset & \text{otherwise}
        \end{cases}
    \]
    Then since $A$ and $\emptyset$ are clearly disjoint, we apply countable additivity
    \[
        m(A) = m \left( \bigcup_{k=1}^{\infty} E_k \right) = \sum_{k=1}^{\infty} m(E_k)
    \]
    This sum diverges since we assumed that $m(\emptyset) > 0$, but this is a contradiction since $m(A) < \infty$.
    Hence, it must be that $m(\emptyset) = 0$.

    \clearpage

    \begin{mybox}
        \textbf{Chapter 2 Problem 3:} \\
        Let $\{ E_k \}_{k=1}^{\infty}$ be a countable collection of sets in $\mathcal{A}$.
        Prove that $m(\cup_{k=1}^{\infty} E_k) \leq \sum_{k=1}^{\infty} m(E_k)$
    \end{mybox}

    If $\{ E_k \}_{k=1}^{\infty}$ is disjoint then
    \[
        m \left( \bigcup_{k=1}^{\infty} E_k \right) = \sum_{k=1}^{\infty} m(E_k)
    \]
    And we are done.
    In the case where $\{ E_k \}_{k=1}^{\infty}$ is not pair-wise disjoint then we define a new set:
    \[
        F_k = \{ e \in E_k \mid e \notin \cup _{i=1}^{\infty} E_i \setminus E_k \}
    \]
    $\{ F_k \}_{k=1}^{\infty}$ is a pair-wise disjoint collection because each $F_k$ has all the elements of $E_k$ except for the ones which are shared in other $E_i$.
    This means that
    \[
        m(F_k) \leq m(E_k) \; \forall k
    \]
    By monotonicity since $F_k \subseteq E_k$.
    Finally, using the fact that we have only removed non-unique values from $\{ E_k \}_{k=1}^{\infty}$ to build $\{ F_k \}_{k=1}^{\infty}$
    \begin{align*}
        m(\bigcup_{k=1}^{\infty} E_k) &= m(\bigcup_{k=1}^{\infty} F_k) = \sum_{k=1}^{\infty} m(F_k) \tag{$F_k$ disjoint} \\
        &\leq \sum_{k=1}^{\infty} m(E_k)
    \end{align*}

    \clearpage

    \begin{mybox}
        \textbf{Chapter 2 Problem 4:} \\
        A set function $c$, defined on all subsets of $\mathbb{R}$, is defined as follows.
        Define $c(E)$ to be $\infty$ if $E$ has infinitely many members and $c(E)$ to be equal to the number of elements in $E$ if $E$ is finite;
        define $c(\emptyset) = 0$.
        Show that $c$ is a countably additive and translation invariant set function.
        This set function is called the \textbf{counting measure}
    \end{mybox}

    To prove \emph{countable additivity}, let $\{ E_k \}_{k=1}^{\infty}$ be a countable collection of disjoint sets of $\mathbb{R}$.
    Since $\{ E_k \}_{k=1}^{\infty}$ is disjoint, by definition $\cup_{k=1}^{\infty} E_k$ will have the same cardinality as the sum of the cardinality of each $E_k$.
    Well this is the same thing as
    \[
        c \left( \bigcup_{k=1}^{\infty} E_k \right) = \sum_{k=1}^{\infty} c(E_k)
    \]
    To prove \emph{translation invariance}, for any set $A \subset \mathbb{R}$ we can create a bijection $f: A \rightarrow A+y$ where $A+y = \{ x+y \mid x \in A \}$ for a given $y \in \mathbb{R}$.
    \[
        f(x) = x+y
    \]
    Let $x \in A + y$ be arbitrary.
    Consider the fact that $x-y \in A$ and $f(x-y) = x - y + y = x$ so $f$ is surjective. \\

    Assume that $f(a) = f(b)$.
    Well then $a + y = b + y \Rightarrow a = b$ and so $f$ is injective.\\

    Since $f$ is injective and surjective, it must be bijective and $\lvert A \rvert = \lvert A+y \rvert$ so
    \[
        c(A) = c(A+y)
    \]


\end{document}