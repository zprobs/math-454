%! Author = zachprobst
%! Date = 2021-11-15

% Preamble
\documentclass[11pt]{article}

% Packages
\usepackage{amsmath}
\usepackage{amsfonts}
\usepackage{amsthm}
\usepackage{mathtools}
\usepackage[makeroom]{cancel}
\usepackage{tcolorbox}

% Setup
\title{Homework 8}
\author{Zachary Probst}
\setlength{\parindent}{0pt}
\setlength{\parskip}{\baselineskip}%
\DeclarePairedDelimiter{\ceil}{\lceil}{\rceil}
\DeclarePairedDelimiter{\card}{\lvert}{\rvert}
\newtcolorbox{mybox}{colback=blue!5!white,colframe=blue!75!black}


\newtheorem{theorem}{Theorem}[section]
\newtheorem*{prop}{Proposition}
\newtheorem{corollary}{Corollary}[theorem]
\newtheorem*{*corollary}{Corollary}
\newtheorem{lemma}[theorem]{Lemma}
\newtheorem{definition}{Definition}[section]
\newtheorem*{remark}{Remark}

% Commands
\newcommand{\siga}{\sigma\text{-algebra}}
\newcommand{\pwr}[1]{\mathcal{P}(\mathbb{#1})}
\newcommand{\rinf}{\mathbb{R}_{\geq 0} \cup \{ + \infty \}}
\newcommand{\mstar}[1]{m^{\star}\left(#1\right)}
\newcommand{\isum}[1]{\sum_{#1=1}^{\infty}}
\newcommand{\pr}{^{\prime}}
\newcommand{\dpr}{^{\prime\prime}}
\newcommand{\gdelta}{G_{\delta}}
\newcommand{\fsigma}{F_{\sigma}}
\newcommand{\R}{\mathbb{R}}
\newcommand{\N}{\mathbb{N}}
\newcommand{\Q}{\mathbb{Q}}


% Document
\begin{document}

    \maketitle

    \begin{mybox}
        \textbf{Problem 1:} Let $E$ be measurable with $m(E) < \infty$ and $f: E \rightarrow \R$ a bounded measurable function.
        Show that we have
        \[
            \card[\Big]{\int_{E} f } = \int_{E} \card{f}
        \]
        if and only if $f \geq 0$ a.e on $E$ or $f \leq 0$ a.e on $E$.
    \end{mybox}

    Assume that $f \geq 0$ a.e on $E$.
    Let $E\pr \coloneqq \{ x \in E : f(x) < 0 \}$.
    Let $c \coloneqq \inf \{ f(x) : x \in E\pr \}$
    Define a function $z: E \rightarrow \{c,0\}$ as
    \[
        z(x) \coloneqq \begin{cases}
                   c & x \in E\pr \\
                   0 & x \notin E\pr
        \end{cases}
    \]
    Since $f \geq 0$ a.e on $E$, we know $m(E\pr) = 0$ so
    \[
        \int_{E} z = \int_{E \setminus E\pr} 0 + \int_{E\pr} c = 0
    \]
    So $z \leq f$ is a simple function and a contender for the supremum in $\mathcal{L}(f)$
    \[
        0 = \int_{E} z \leq \mathcal{L}(f) = \int_{E} f
    \]
    Now we know that $\int_{E} f \geq 0$ so
    \[
        \card[\Big]{\int_{E} f } = \int_{E} f = \int_{E \setminus E\pr} f + \int_{E\pr} f = \int_{E \setminus E\pr} f = \int_{E \setminus E\pr} \card{f} =  \int_{E} \card{f}
    \]

    Now assume that $f \leq 0$ a.e on $E$.
    Then $-f \geq 0$ a.e on $E$ and we have already proved that
    \[
        \card[\Big]{\int_{E} -f} = \int_{E} \card{-f} = \int_{E} \card{f}
    \]
    But the lebesgue integral satisfies linearity so
    \[
        \card[\Big]{\int_{E} -f} = \card[\Big]{- \int_{E} f } = \card[\Big]{\int_{E} f}
    \]
    And putting that together we get
    \[
        \card[\Big]{\int_{E} f} = \int_{E} \card{f}
    \]

    Now assume that $\card[\Big]{\int_{E} f} = \int_{E} \card{f}$.
    Let $E\pr \coloneqq \{ x \in E : f(x) > 0 \}$ and $E\dpr \coloneqq \{ x \in E : f(x) < 0 \}$.
    By way of contradiction, let us assume that $m(E\pr) > 0$ and $m(E \dpr) > 0$.
    $f = 0$ on $E \setminus E\pr \cup E\dpr$ so we can decompose the integral of $f$ like so:
    \[
        \card[\Big]{\int_{E} f} = \int_{E} \card{f} = \int_{E\pr} \card{f} + \int_{E\dpr} \card{f} = \card[\Big]{\int_{E\pr} f} + \card[\Big]{\int_{E\dpr} f}
    \]
    But
    \[
        \card[\Big]{\int_{E} f} = \card[\Big]{\int_{E\pr} f + \int_{E\dpr} f} = \card[\Big]{\int_{E\pr} f} + \card[\Big]{\int_{E\dpr} f}
    \]
    Which is only possible if both $\int_{E\pr} f \geq 0$ and $\int_{E\dpr} f \geq 0$ or if one term were equal to 0.
    But $\int_{E\dpr} f < 0 $ and $\int_{E\pr} f > 0 $ so we have a contradiction.

    So it must be the case that either $m(E\pr) = 0$ or $m(E\dpr) = 0$.
    This means that either $f \leq 0$ a.e on $E$ or $f \geq 0$ a.e on $E$.

    \clearpage

    \begin{mybox}
        \textbf{Chapter 4 Problem 26:} Show that the Monotone Convergence Theorem may not hold for decreasing sequences of functions.
    \end{mybox}

    Let $E \coloneqq [0,\infty)$.
    Define $f_n: E \rightarrow \R$ as $f_n (x) = \frac{1}{n}$.
    Clearly $f_n$ is a decreasing sequence of functions that converges pointwise to the zero function, since $\lim_{n \rightarrow \infty} \frac{1}{n} = 0$.
    However, we can write each $f_n$ as a simple function whose integral is
    \[
        \int_{E} f_n = \int_{[0,1)} f_n + \int _{[1,2)} f_n + \hdots = \sum_{i=1}^{\infty} \frac{1}{n} \geq \sum_{i=1}^{n} \frac{1}{n} = 1
    \]
    Letting $n \rightarrow \infty$ we get
    \[
        \lim_{n \rightarrow \infty} \int_{E} f_n \geq 1
    \]
    But $\int_{E} 0 = 0$, therefore the Monotone Convergence Theorem may not hold for decreasing sequences of functions.

    \clearpage

    \begin{mybox}
        \textbf{Chapter 4 Problem 27} Prove the following estimation of Fatou's Lemma: If $\{ f_n \}$ is a sequence of non-negative measurable functions on $E$, then
        \[
            \int_{E} \liminf f_n \leq \liminf \int_{E} f_n
        \]
    \end{mybox}

    Let $N \in \N$, define $c \coloneqq \inf\{ f_n : n \geq N \}$ then $c \leq f_n$ on $E$ $\forall n \geq N$.
    Let $\phi_N: E \rightarrow \{ c \}$ be a simple function on $E$ with $\phi_N \leq f_n \; \forall n \geq N$ so
    \[
        \int_{E} \phi_N \leq \int_{E} f_n
    \]
    $\int_{E} \phi_N$ is a lower bound for $\{ \int_{E} f_n : n \geq N \}$ hence
    \[
        \int_{E} \phi_N \leq \inf \left\{  \int_{E} f_n : n \geq N \right\}
    \]
    $\phi_N$ converges pointwise to $\liminf f_n$ and $0 \leq \phi_N \leq \liminf f_n$ so $\card{\phi_N} \leq \liminf f_n$ and we can apply the Lebesgue Dominated Convergence Theorem.
    \[
        \int_{E} \liminf f_n = \lim_{N \rightarrow \infty} \int_{E} \phi_N  \leq \lim_{N \rightarrow \infty} \inf \left\{  \int_{E} f_n : n \geq N \right\} \right\}  = \liminf \int_{E} f_n
    \]

\end{document}