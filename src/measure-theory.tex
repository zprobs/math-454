We want to measure the size of a set.
We will deal with a subset of $\mathbb{R}$.

It turns out that one needs to select a class of subsets of $\mathbb{R}$ that one wants to measure.
This class of subsets will have certain properties which are as follows.

\begin{definition}[$\siga$]
    A collection $\mathcal{A}$ of subsets of $\mathbb{R}$ is called a $\siga$ if it satisfies
    \begin{enumerate}
        \item $\emptyset \in \mathcal{A}$
        \item If $A \in \mathcal{A}$ then $A^c \in \mathcal{A}$
        \item If $\{ A_n \}_{n=1}^{\infty} \subset \mathcal{A}$ then $\cup_{n=1}^{\infty} A_n \in \mathcal{A}$
    \end{enumerate}
\end{definition}

Observe the following:
\begin{itemize}
    \item $\mathbb{R} \in \mathcal{A}$ always
    \item If $\{ A_n \}_{n=1}^{N} \subset \mathcal{A}$ then $\cup_{n=1}^{N} A_n \in \mathcal{A}$ (just define $A_n = \emptyset$ for $n > N$)
    \item If $\{ A_n \}_{n=1}^{\infty} \subset \mathcal{A}$ then $\cap_{n=1}^{\infty} A_n \in \mathcal{A}$ (since $(\cap_{n=1}^{\infty} A_n)^c = \cup_{n=1}^{\infty} A_{n}^{c}$)
    \item If $A,B \in \mathcal{A}$ then $A \setminus B \in \mathcal{A}$ too since $A \setminus B = A \cap B^c$
\end{itemize}

\underline{\textbf{Examples}:}

\begin{enumerate}
    \item $\mathcal{A} = \{ \emptyset, \mathbb{R} \}$ ``Minimal $\siga$''
    \item $\mathcal{A} = \mathcal{P}(\mathbb{R}) = $ Collection of all subsets of $\mathbb{R}$.
    ``Maximum $\siga$''
\end{enumerate}

In fact, if $\mathcal{A}$ is any $\siga$, then $\{ \emptyset,\mathbb{R} \} \subseteq \mathcal{A} \subseteq \mathcal{P}(\mathbb{R})$

For better examples, let $F$ be any collection of subsets of $\mathbb{R}$.
I want to make $F$ into a $\siga$.
Define $m = \{ \mathcal{A} \mid \mathcal{A} \text{ is a } \siga \text{ that satisfies } F \subset \mathcal{A} \}$.
$m \neq \emptyset$ since it contains $\mathcal{P}(\mathbb{R})$

If $\mathcal{A}, \mathcal{B} \in m$, I can define $\mathcal{A} \cap \mathcal{B} = \{ A \subset \mathbb{R} \mid A \in \mathcal{A} \text{ and } A \in \mathcal{B} \}$ and I can do the same for $\cap_{i \in I} \mathcal{A}$ arbitrary intersection of $\siga$ is still a $\siga$

Define $\hat{F_i} = \cap_{\mathcal{A} \in m} \mathcal{A}$ as a $\siga$ and $F \subset \hat{F}$ and it is the minimal $\siga$ with these properties.
If $G$ is a $\siga$ with $F \subset G$, then $\hat{F} \subset G$.
$\hat{F}$ is the $\siga$ generated by $F$.
Concretely, $\hat{F}$ consists of all subsets of $\mathbb{R}$ that can be constructed by applying countable unions, intersections, and complements to elements of $F$.

\begin{definition}[Borel Sets]
    The $\siga \; \mathcal{B}$ of Borel Sets is the $\siga \; \hat{F}$ generated by
    \[
        F = \{ U \subset \mathbb{R} \mid U \text{ open } \}
    \]
\end{definition}

\begin{remark}
    $\mathcal{B}$ is also the $\siga$ generated by the family of all closed subsets of $\mathbb{R}$
\end{remark}

Singletons $\{ x \} \subset \mathbb{R}$ are closed so if $A \subset \mathbb{R}$ is at most countable then $A$ is Borel.
(e.g $\mathbb{Q} \subset \mathbb{R}$) (e.g $\mathbb{R} \setminus \mathbb{Q}$)

Not all Subsets of $\mathbb{R}$ are Borel.
One can actually show that the cardinality of $\mathcal{B}$ is the same as the cardinality of $\mathbb{R}$.
On the other hand $\mathcal{P}(\mathbb{R})$ has strictly larger cardinality.